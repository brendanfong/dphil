\chapter[Hypergraph categories: the algebra of interconnection]{Hypergraph
categories: the algebra of interconnection}

\section{The algebra of interconnection}

Our aim is to algebraicise network diagrams. A network diagram is built from
pieces like so:
\[
  \begin{tikzpicture}
    \node [draw,circle,thick] (s) at (0,0) {};
    \node (a) at (-2,0) {};
    \node (b) at (1,1.7) {};
    \node (c) at (1,-1.7) {};
    \draw (s) to (a);
    \draw [dashed] (s) to (b);
    \draw (s) to (c);
  \end{tikzpicture}
\]
These represent open systems, concrete or abstract; for example a resistor, a
chemical reaction, or a linear transformation. The essential feature, for
openness and for networking, is that the system may have terminals, perhaps of
different `types', each one depicted by a line radiating from the central body.
In the case of a resistor each terminal might represent a wire, for chemical
reactions a chemical species, for linear transformations a variable in the
domain or codomain.  Network diagrams are formed by connecting terminals of
systems to build larger systems.

A network-style diagrammatic language is a collection of network diagrams
together with the stipulation that if we take some of these network diagrams,
and connect terminals of the same type in any way we like, then we form
another diagram in the collection.  The point of this chapter is that hypergraph
categories provide a precise formalisation of network-style diagrammatic
languages.  

\begin{figure}
\[
  \begin{tikzpicture}
    \node [draw,circle,thick] (s) at (0,0) {};
    \node (a) at (-2,0) {};
    \node (b) at (1,1.7) {};
    \node (c) at (1,-1.7) {};
    \draw (s) to (a);
    \draw (s) to (b);
    \draw (s) to (c);
  \end{tikzpicture}
  \begin{tikzpicture}
    \node [draw,circle,thick] (s) at (0,0) {};
    \node (a) at (-2,0) {};
    \node (b) at (1,1.7) {};
    \node (c) at (1,-1.7) {};
    \draw (s) to (a);
    \draw (s) to (b);
    \draw (s) to (c);
  \end{tikzpicture}
\]
\caption{Interconnection of network diagrams. Note that we only connect
terminals of the same type, but we can connect as many as we like.}
\end{figure}

In jargon, a hypergraph category is a symmetric monoidal category in
which every object is equipped with a special commutative Frobenius monoid in a
way compatible with the monoidal product. We will walk through these terms in
detail, illustrating them with examples and a few theorems. 

The key data comprising a hypergraph category are its objects, morphisms,
composition rule, monoidal product, and Frobenius maps. Each of these model a
feature of network diagrams and their interconnection. The objects model the
terminal types, while the morphisms model the network diagrams themselves. The
composition, monoidal product, and Frobenius maps model different aspects of
interconnection: composition models the interconnection of two terminals of the
same type, the monoidal product models the network formed by taking two networks
without interconnecting any terminals, while the Frobenius maps model
multi-terminal interconnection.

These Frobenius maps are the distinguishing feature of hypergraph categories as
compared to other structured monoidal categories, and are crucial for
formalising the intuitive concept of network languages detailed above. In the
case of electric circuits the Frobenius maps model the `branching' of wires; in
the case when diagrams simply model an abstract system of equations and
terminals variables in these equations, the Frobenius maps allow variables to be
shared between many systems of equations.

Examining these correspondences, it is worthwhile to ask whether hypergraph
categories permit too much structure to be specified, given that the
interconnection rule is now divided into three different aspects, and features
such as domains and codomains of network diagrams, rather than just a collection
of terminals, exist. The answer is given by examining the additional coherence
laws that these data must obey. For example, in the case of the domain and
codomain, we shall see that hypergraph categories are all compact closed
categories, and so there is ultimately only a formal distinction between domain
and codomain objects. One way to think of these data is as scaffolding. We could
compare it to the use of matrices and bases to provide language for talking
about linear transformations and vector spaces.  They are not part of the target
structure, but nonetheless useful paraphenalia for constructing it.

\begin{figure}
  \begin{center}
  \begin{tabular}{c|c}
    Networks & Hypergraph categories \\
    \hline 
    list of terminal types & object \\
    network diagram & morphism \\
    series connection & composition \\
    juxtaposition & monoidal product \\
    branching & Frobenius maps
  \end{tabular}
  \end{center}
  \caption{Corresponding features of networks and hypergraph categories.}
\end{figure}

Network languages are not only syntactic entities: as befitting the descriptor
`language', they typically have some associated semantics. Circuits diagrams, for
instance, not only depict wire circuits that may be constructed, they also
represent the electrical behaviour of that circuit. Such semantics considers the
circuits 
\[
  \begin{aligned}
  \begin{tikzpicture}[circuit ee IEC, set resistor graphic=var resistor IEC graphic]
    \node[contact] (I1) at (0,0) {};
    \node[circle, minimum width = 3pt, inner sep = 0pt, fill=black] (int) at (3,0) {};
    \node[contact] (O1) at (6,0) {};
    \draw (I1) 	to [resistor] node [label={[label distance=3pt]90:{$1 \Omega$}}] {} (int)
    to [resistor] node [label={[label distance=3pt]90:{$1 \Omega$}}] {} (O1);
  \end{tikzpicture}
  \end{aligned}
  \qquad
  \mbox{and}
  \qquad
  \begin{aligned}
  \begin{tikzpicture}[circuit ee IEC, set resistor graphic=var resistor IEC graphic]
    \node[contact] (I1) at (0,0) {};
    \node[contact] (O1) at (3,0) {};
    \draw (I1) 	to [resistor] node [label={[label distance=3pt]90:{$2 \Omega$}}]
    {} (O1);
  \end{tikzpicture}
  \end{aligned}
\]
the same, even though as `syntactic' diagrams they are distinct. A cornerstone
of the utility of the hypergraph formalism is the ability to also realise the
semantics of these diagrams as morphisms of another hypergraph category. This
`semantic' hypergraph category, as a hypergraph category, still permits the rich
`networking' interconnection structure, and a so-called hypergraph functor
implies that the syntactic category provides a sound framework for depicting
these morphisms. Network languages syntactically are often `free' hypergraph
categories, and much of the interesting structure lies in their functors to
their semantic hypergraph categories.

In this chapter we introduce hypergraph categories, giving a definition,
coherence theorem, and graphical language. We then explore a fundamental example
of hypergraph categories: categories of cospans.

We assume basic familiarity with category theory and symmetric monoidal
categories; although we give a sparse overview of the latter for reference. A
proper introduction to both can be found in Mac Lane \cite{Mac98}.

\section{Symmetric monoidal categories}
Hypergraph categories are first symmetric monoidal categories. Moreover,
symmetric monoidal functors play a key role in our framework for defining and
working with hypergraph categories: decorated cospans and corelations
constructions. For this reason we provide, for reference, a definition of
symmetric monoidal categories and their morphisms. 

A symmetric monoidal category is a category with two notions of composition:
ordinary categorical composition and monoidal composition, with the monoidal
composition only associative and unital up to natural isomorphism. They are the
algebra of processes that may occur simultaneously as well as sequentially.
Relatedly, they have an associated graphical calculus, which strongly motivates
their use here in the formalisation of network languages. 

\subsection{String diagrams}
Suppose we have some tiles with inputs and outputs of various types like so:
\[
  .
\]

\subsection{Monoidal categories}
A \define{monoidal category} $(\c, \ot)$ consists of a category $\c$, a
functor $\ot: \c \times \c \to \c$, a distinguished object $I$, and natural
isomorphisms $\a_{A,B,C}: (A \ot B) \ot C \to A \ot (B \ot C)$,
$\rho_A: A \ot I  \to A$, and $\lambda_A: I \ot A \to A$ such that for all
$A,B,C,D$ in $\mc C$ the following two diagrams commute: 
\[
  \xymatrixcolsep{3pc}
  \xymatrix{
    \big((A \ot B) \ot C\big) \ot D \ar[d]_{\a_{A,B,C}\ot\idn_D} \ar[rr]^{\a_{(A\ot B),C,D}} 
    &&(A \ot B) \ot (C \ot D) \ar[d]^{\a_{A,B,(C\ot D)}} \\
    \big(A \ot (B\ot C)\big) \ot D \ar[r]_{\a_{A,(B\ot C),D}} 
    & A\ot\big((B \ot C)\ot D\big)\ar[r]_{\idn_A \ot \a_{B,C,D}}
    &A \ot \big(B \ot (C \ot D)\big)
  }
\]
\[
  \xymatrix{
    (A\ot I)\ot B  \ar[rr]^{\a_{A,I,B}} \ar[dr]_{\rho_{A}\ot \idn_B} && A \ot (I \ot B) \ar[dl]^{\idn_A\ot \lambda_B}\\
    & A \ot B \\
  }
\]
We call $\ot$ the \define{monoidal product}, $I$ the \define{monoidal unit},
$\alpha$ the \define{associator}, $\rho$ and $\lambda$ the \define{right} and
\define{left unitor} respectively. The associator and unitors are known
collectively as the \define{coherence maps}.

By Mac Lane's coherence theorem, these two axioms are equivalent to requiring
that `all formal diagrams'---that is, all diagrams in which the morphism are
built from identity morphisms and the coherence maps using composition and the
monoidal product---commute. Consequently, between any two products of the same
ordered list of objects up to instances of the monoidal unit, such as $((A \ot
I) \ot B) \ot C$ and $A \ot ((B \ot C) \ot (I \ot I))$, there is a unique
so-called \define{canonical} map. See Mac Lane \cite[Corollary of Theorem
VII.2.1]{Mac98} for a precise statement and proof.

A \define{lax monoidal functor} $(F, \varphi): (\c,\otimes) \to (\c',\boxtimes)$
between monoidal categories consists of a functor $F: \c \to \c'$, and natural
transformations $\varphi_{A,B}: FA \boxtimes FB \to F(A \ot B)$ and $\varphi_1:
1_{\c'} \to F1_{\c}$, such that for all $A,B,C \in \c$ the three diagrams
\[
  \xymatrixcolsep{4pc}
  \xymatrix{
    (FA \ot FB) \ot FC \ar[d]_{\a_{FA,FB,FC}} \ar[r]^{\varphi_{A,B} \ot \idn_{FC}} &
    F(A \ot B) \ot FC \ar[r]^{\varphi_{A\ot B,C}} & F((A \ot B) \ot C) \ar[d]^{F\a_{A,B,C}}\\
    FA \ot (FB \ot FC) \ar[r]_{\idn_{FA} \ot \varphi_{B,C}} & FA \ot F(B \ot C)
    \ar[r]_{\varphi_{A,B\ot C}} & F(A \ot (B \ot C))
  }
\]
\[
  \xymatrixcolsep{3pc}
  \xymatrixrowsep{3pc}
  \xymatrix{
    F(A) \ot I' \ar[d]_{\idn \ot \varphi_1} \ar[r]^{\rho} & F(A) \\
    F(A) \ot F(I) \ar[r]_{\varphi_{A,I}} & F(A \ot I) \ar[u]_{F\rho} 
  }
  \qquad
  \xymatrix{
    I' \ot F(A) \ar[d]_{\varphi_1 \ot \idn} \ar[r]^{\lambda} & F(A) \\
    F(I) \ot F(A) \ar[r]_{\varphi_{I,A}} & F(I \ot A) \ar[u]_{F\lambda} 
  }
\]
commute. We further say a monoidal functor is a \define{strong monoidal functor}
if the $\varphi$ are isomorphisms, and a \define{strict monoidal functor} if the
$\varphi$ are identities. 

A \define{monoidal natural transformation} $\theta: (F,\varphi) \Rightarrow
(G,\gamma)$ between two monoidal functors $F$ and $G$ is a natural
transformation $\theta: F \Rightarrow G$ such that
\[
  \begin{aligned}
    \xymatrix{
      F1_{\c} \ar[rr]^{\theta_I}&& G1_{\c} \\
      & 1_{\c'} \ar[ul]^{\varphi_1} \ar[ur]_{\gamma_1}
    } 
  \end{aligned} 
  \qquad 
  \mbox{and}
  \qquad
  \begin{aligned}
    \xymatrixcolsep{3pc}
    \xymatrixrowsep{3pc}
    \xymatrix{
      FA \boxtimes FB \ar[r]^{\theta_A \ot \theta_B} \ar[d]_{\varphi_{A,B}} 
      & GA \boxtimes GB \ar[d]^{\gamma_{A,B}}\\
      F(A \ot B) \ar[r]_{\theta_{A\ot B}} & G(A \ot B)
    }
  \end{aligned} 
\]
commute for all objects $A,B$.

Two monoidal categories $\mc C, \mc D$ are \define{monoidally equivalent} if
there exist strong monoidal functors $F\maps \mc C \to \mc D$ and $G\maps \mc D
\to \mc C$ such that the composites $FG$ and $GF$ are monoidally naturally
isomorphic to the identity functors. (Note that identity functors are
immediately strict monoidal functors.)

\subsection{Coherence}
A \define{strict monoidal category} category is a monoidal category in which the
associators and unitors are all identity maps. In this case then any two objects
that can be related by associators and unitors are equal, and so we may write
objects without parentheses and units without ambiguity. An equivalent statement
of Mac Lane's coherence theorem is that every symmetric monoidal category is
equivalent as a symmetric monoidal category to strict symmetric monoidal category. 

Yet another equivalent statement of the coherence theorem is the existence of a
graphical calculus for morphisms in symmetric monoidal categories. Monoidal
categories figure strongly in our current investigations precisely because of
this. We leave the details to discussions elsewhere. The main point is that we
shall be free to assume our symmetric monoidal categories are strict, writing
$X_1 \otimes X_2 \otimes \dots \otimes X_n$ for objects in $(\mathcal
C,\otimes)$ without a care for parentheses. We then depict a morphism $f\maps
X_1 \otimes X_2 \otimes \dots \otimes X_n \to Y_1 \otimes Y_2 \otimes \dots
\otimes Y_n$ with the diagram:
\[
  f \quad = \quad
  \begin{aligned}
    \tikzset{every path/.style={line width=1.1pt}}
  \begin{tikzpicture}
	\begin{pgfonlayer}{nodelayer}
		\node [style=none] (0) at (-0.25, 0.375) {};
		\node [style=none] (1) at (0.5, 0.375) {};
		\node [style=none] (2) at (-0.25, -1.375) {};
		\node [style=none] (3) at (0.5, -1.375) {};
		\node [style=none] (4) at (0.5, 0.25) {};
		\node [style=none] (5) at (0.5, -1.25) {};
		\node [style=none] (6) at (1.25, 0.25) {};
		\node [style=none] (7) at (1.25, -1.25) {};
		\node [style=none] (8) at (0.125, -0.5) {$f$};
		\node [style=none] (9) at (1.5, 0.25) {$Y_1$};
		\node [style=none] (10) at (1.5, -1.25) {$Y_m$};
		\node [style=none] (11) at (1.25, -0.25) {};
		\node [style=none] (12) at (1.5, -0.25) {$Y_2$};
		\node [style=none] (13) at (0.5, -0.25) {};
		\node [style=none] (14) at (1, -0.75) {$\vdots$};
		\node [style=none] (15) at (-1, -1.25) {};
		\node [style=none] (16) at (-0.25, -1.25) {};
		\node [style=none] (17) at (-0.75, -0.75) {$\vdots$};
		\node [style=none] (18) at (-1.25, -1.25) {$X_n$};
		\node [style=none] (19) at (-0.25, -0.25) {};
		\node [style=none] (20) at (-1.25, 0.25) {$X_1$};
		\node [style=none] (21) at (-1, 0.25) {};
		\node [style=none] (22) at (-0.25, 0.25) {};
		\node [style=none] (23) at (-1, -0.25) {};
		\node [style=none] (24) at (-1.25, -0.25) {$X_2$};
	\end{pgfonlayer}
	\begin{pgfonlayer}{edgelayer}
		\draw (0.center) to (1.center);
		\draw (1.center) to (3.center);
		\draw (3.center) to (2.center);
		\draw (2.center) to (0.center);
		\draw (4.center) to (6.center);
		\draw (5.center) to (7.center);
		\draw (13.center) to (11.center);
		\draw (22.center) to (21.center);
		\draw (16.center) to (15.center);
		\draw (19.center) to (23.center);
	\end{pgfonlayer}
\end{tikzpicture}.
\end{aligned}
\]
Identity morphisms are depicted by `wires':
\[
  \idn_X \quad = \quad
  \begin{aligned}
    \tikzset{every path/.style={line width=1.1pt}}
\begin{tikzpicture}
	\begin{pgfonlayer}{nodelayer}
		\node [style=none] (0) at (1.25, 0.25) {};
		\node [style=none] (1) at (1.5, 0.25) {$X$};
		\node [style=none] (2) at (-1.25, 0.25) {$X$};
		\node [style=none] (3) at (-1, 0.25) {};
	\end{pgfonlayer}
	\begin{pgfonlayer}{edgelayer}
		\draw (3.center) to (0.center);
	\end{pgfonlayer}
\end{tikzpicture}
\end{aligned}
\]
and the monoidal unit is not depicted at all:
\[
\idn_I\quad = \quad
  \begin{aligned}
    \tikzset{every path/.style={line width=1.1pt}}
\begin{tikzpicture}
		\node [style=none] (1) at (1.5, 0.25) {};
		\node [style=none] (2) at (-1.25, 0.25) {};
\end{tikzpicture}
\end{aligned}
\]
Composition of morphisms is depicted by connecting the relevant `wires'
\[
    \tikzset{every path/.style={line width=1.1pt}}
  \begin{aligned}
    \begin{tikzpicture}
	\begin{pgfonlayer}{nodelayer}
		\node [style=none] (0) at (0.25, -0) {$Y_1$};
		\node [style=none] (1) at (0.5, -0) {};
		\node [style=none] (2) at (2.75, -0.75) {};
		\node [style=none] (3) at (3, -0.75) {$Y_2$};
		\node [style=none] (4) at (0.5, -0.75) {};
		\node [style=none] (5) at (2, -0.365) {};
		\node [style=none] (6) at (2.75, 0.25) {};
		\node [style=none] (7) at (3, 0.25) {$Z_1$};
		\node [style=none] (8) at (1.25, 0.375) {};
		\node [style=none] (9) at (1.25, -0) {};
		\node [style=none] (10) at (1.25, -0.365) {};
		\node [style=none] (11) at (2, 0.25) {};
		\node [style=none] (12) at (2, 0.375) {};
		\node [style=none] (13) at (1.625, -0) {$g$};
		\node [style=none] (14) at (2, -0.25) {};
		\node [style=none] (15) at (2.75, -0.25) {};
		\node [style=none] (16) at (3, -0.25) {$Z_2$};
		\node [style=none] (17) at (0.25, -0.75) {$Y_2$};
	\end{pgfonlayer}
	\begin{pgfonlayer}{edgelayer}
		\draw (4.center) to (2.center);
		\draw (8.center) to (12.center);
		\draw (5.center) to (10.center);
		\draw (11.center) to (6.center);
		\draw (12.center) to (5.center);
		\draw (10.center) to (8.center);
		\draw (14.center) to (15.center);
		\draw (1.center) to (9.center);
	\end{pgfonlayer}
\end{tikzpicture}
\end{aligned}
  \circ
  \begin{aligned}
  \begin{tikzpicture}
	\begin{pgfonlayer}{nodelayer}
		\node [style=none] (0) at (-0.25, 0.375) {};
		\node [style=none] (1) at (0.5, 0.375) {};
		\node [style=none] (2) at (-0.25, -.875) {};
		\node [style=none] (3) at (0.5, -.875) {};
		\node [style=none] (4) at (0.5, 0.125) {};
		\node [style=none] (5) at (1.25, 0.125) {};
		\node [style=none] (6) at (0.125, -0.25) {$f$};
		\node [style=none] (7) at (1.5, 0.125) {$Y_1$};
		\node [style=none] (8) at (1.25, -0.625) {};
		\node [style=none] (9) at (1.5, -0.625) {$Y_2$};
		\node [style=none] (10) at (0.5, -0.625) {};
		\node [style=none] (11) at (-0.25, -0.75) {};
		\node [style=none] (12) at (-1.25, -0.75) {$X_3$};
		\node [style=none] (13) at (-0.25, -0.25) {};
		\node [style=none] (14) at (-1.25, 0.25) {$X_1$};
		\node [style=none] (15) at (-1, 0.25) {};
		\node [style=none] (16) at (-0.25, 0.25) {};
		\node [style=none] (17) at (-1, -0.25) {};
		\node [style=none] (18) at (-1.25, -0.25) {$X_2$};
		\node [style=none] (19) at (-1, -0.75) {};
	\end{pgfonlayer}
	\begin{pgfonlayer}{edgelayer}
		\draw (0.center) to (1.center);
		\draw (1.center) to (3.center);
		\draw (3.center) to (2.center);
		\draw (2.center) to (0.center);
		\draw (4.center) to (5.center);
		\draw (10.center) to (8.center);
		\draw (16.center) to (15.center);
		\draw (13.center) to (17.center);
		\draw (11.center) to (19.center);
	\end{pgfonlayer}
\end{tikzpicture}
\end{aligned}
\quad = \quad
\begin{aligned}
\begin{tikzpicture}
	\begin{pgfonlayer}{nodelayer}
		\node [style=none] (0) at (-0.25, 0.375) {};
		\node [style=none] (1) at (0.5, 0.375) {};
		\node [style=none] (2) at (-0.25, -0.875) {};
		\node [style=none] (3) at (0.5, -0.875) {};
		\node [style=none] (4) at (0.5, 0.125) {};
		\node [style=none] (5) at (0.125, -0.25) {$f$};
		\node [style=none] (6) at (2.75, -0.625) {};
		\node [style=none] (7) at (3, -0.625) {$Y_2$};
		\node [style=none] (8) at (0.5, -0.625) {};
		\node [style=none] (9) at (-0.25, -0.75) {};
		\node [style=none] (10) at (-1.25, -0.75) {$X_3$};
		\node [style=none] (11) at (-0.25, -0.25) {};
		\node [style=none] (12) at (-1.25, 0.25) {$X_1$};
		\node [style=none] (13) at (-1, 0.25) {};
		\node [style=none] (14) at (-0.25, 0.25) {};
		\node [style=none] (15) at (-1, -0.25) {};
		\node [style=none] (16) at (-1.25, -0.25) {$X_2$};
		\node [style=none] (17) at (-1, -0.75) {};
		\node [style=none] (18) at (2, -0.25) {};
		\node [style=none] (19) at (2.75, 0.375) {};
		\node [style=none] (20) at (3, 0.375) {$Z_1$};
		\node [style=none] (21) at (1.25, 0.5) {};
		\node [style=none] (22) at (1.25, 0.125) {};
		\node [style=none] (23) at (1.25, -0.25) {};
		\node [style=none] (24) at (2, 0.375) {};
		\node [style=none] (25) at (2, 0.5) {};
		\node [style=none] (26) at (1.625, 0.125) {$g$};
		\node [style=none] (27) at (2, -0.125) {};
		\node [style=none] (28) at (2.75, -0.125) {};
		\node [style=none] (29) at (3, -0.125) {$Z_2$};
	\end{pgfonlayer}
	\begin{pgfonlayer}{edgelayer}
		\draw (0.center) to (1.center);
		\draw (1.center) to (3.center);
		\draw (3.center) to (2.center);
		\draw (2.center) to (0.center);
		\draw (8.center) to (6.center);
		\draw (14.center) to (13.center);
		\draw (11.center) to (15.center);
		\draw (9.center) to (17.center);
		\draw (21.center) to (25.center);
		\draw (18.center) to (23.center);
		\draw (24.center) to (19.center);
		\draw (25.center) to (18.center);
		\draw (23.center) to (21.center);
		\draw (27.center) to (28.center);
		\draw (4.center) to (22.center);
	\end{pgfonlayer}
\end{tikzpicture}
\end{aligned}
\]
while monoidal composition is just juxtaposition
\[
 .
\]
For example, a diagram such as
\[
  .
\]
reads as the equivalent algebraic expressions 
\[
  .
\]
and so on.

These two theorems show that the graphical calculi go beyond visualisations of the morphisms, having the ability to provide bona-fide proofs of equalities of morphisms. As a general principle, one which we shall demonstrate in this dissertation, this fact combined the intuitiveness of manipulations and the encoding of certain equalities and structural isomorphisms make the string diagrams better than the conventional algebraic language for understanding monoidal categories.

\subsection{Symmetry}

  isomorphisms $\s_{A,B}: A \ot B \to B \ot A$ natural in $A$ and $B$
    such that $\s_{B,A} \circ \s_{A,B} = \idn_{A\ot B}$ called the
    \define{braiding}

    category (note we dropped associators)
\[
\xymatrixcolsep{4pc}
  \xymatrix{
    (A \ot B) \ot C \ar[d]_{\a_{A,B,C}} \ar[r]^{\s_{A,B} \ot \idn_C} & (B \ot A) \ot
    C \ar[r]^{\a_{B,A,C}} & B \ot (A \ot C) \ar[d]^{\idn_B \ot \s_{A,C}}\\
    A \ot (B \ot C) \ar[r]_{\s_{A,B \ot C}} & (B \ot C) \ot A \ar[r]_{\a_{B,C,A}} & B \ot (C \ot A)
  }
\]

functor
\[
\xymatrixcolsep{3pc}
\xymatrixrowsep{3pc}
\xymatrix{
FA \ot FB \ar[r]^{\varphi_{A,B}} \ar[d]_{\s'_{FA,FB}} & F(A \ot B)\ar[d]^{F\s_{A,B}}\\
FB \ot FA \ar[r]_{\varphi_{B,A}} & F(B \ot A)
}
\]

In a symmetric monoidal category we represent the braiding with a special notation, the crossing of two wires:
\[
\s_{A,B} =
\begin{aligned}
  \swap{.1\textwidth}
\end{aligned}
\]
We will also later introduce similar special notations for the Frobenius maps in
a hypergraph category.


The defining identities of the swap may then be written graphically as
\[
\begin{aligned}
\begin{tikzpicture}
	\begin{pgfonlayer}{nodelayer}
		\node [style=none] (0) at (-0.5, -1.25) {};
		\node [style=none] (1) at (0.5, -1.25) {};
		\node [style=none] (2) at (-0.5, 0) {};
		\node [style=none] (3) at (0.5, 0) {};
		\node [style=none] (4) at (0.5, -1) {};
		\node [style=none] (5) at (-0.5, -1) {};
		\node [style=none] (6) at (-0.5, -1.5) {$A$};
		\node [style=none] (7) at (0.5, -1.5) {$B$};
		\node [style=none] (8) at (0.5, 0) {};
		\node [style=none] (9) at (0.5, 1) {};
		\node [style=none] (10) at (-0.5, 1.5) {$A$};
		\node [style=none] (11) at (0.5, 1.5) {$B$};
		\node [style=none] (12) at (-0.5, 1.25) {};
		\node [style=none] (13) at (-0.5, 0) {};
		\node [style=none] (14) at (-0.5, 1) {};
		\node [style=none] (15) at (0.5, 1.25) {};
	\end{pgfonlayer}
	\begin{pgfonlayer}{edgelayer}
		\draw [in=90, out=-90, looseness=0.75] (2.center) to (4.center);
		\draw (4.center) to (1.center);
		\draw [in=90, out=-90, looseness=0.75] (3.center) to (5.center);
		\draw (5.center) to (0.center);
		\draw (12.center) to (14.center);
		\draw [in=90, out=-90, looseness=0.75] (14.center) to (8.center);
		\draw (15.center) to (9.center);
		\draw [in=90, out=-90, looseness=0.75] (9.center) to (13.center);
	\end{pgfonlayer}
\end{tikzpicture}
\end{aligned}
\quad
=
\quad
\begin{aligned}
\begin{tikzpicture}
	\begin{pgfonlayer}{nodelayer}
		\node [style=none] (0) at (-0.5, -1.25) {};
		\node [style=none] (1) at (0.5, -1.25) {};
		\node [style=none] (2) at (-0.5, -1.5) {$A$};
		\node [style=none] (3) at (0.5, -1.5) {$B$};
		\node [style=none] (4) at (-0.5, 1.5) {$A$};
		\node [style=none] (5) at (0.5, 1.5) {$B$};
		\node [style=none] (6) at (-0.5, 1.25) {};
		\node [style=none] (7) at (0.5, 1.25) {};
	\end{pgfonlayer}
	\begin{pgfonlayer}{edgelayer}
		\draw (6.center) to (0.center);
		\draw (7.center) to (1.center);
	\end{pgfonlayer}
\end{tikzpicture}
\end{aligned}\label{swapdiag1}\tag{Sym1}
\]
and
\[\label{swapdiag2}\tag{Sym2}
\begin{aligned}
\begin{tikzpicture}
	\begin{pgfonlayer}{nodelayer}
		\node [style=none] (0) at (-0.5, -0.25) {};
		\node [style=none] (1) at (0.5, -0.25) {};
		\node [style=none] (2) at (-0.5, 1.5) {};
		\node [style=none] (3) at (0.5, 0.5) {};
		\node [style=none] (4) at (-0.5, 0.5) {};
		\node [style=none] (5) at (-0.5, -0.5) {$A$};
		\node [style=none] (6) at (0.5, -0.5) {$B$};
		\node [style=none] (7) at (1.5, 0.5) {};
		\node [style=none] (8) at (0.5, 2.5) {$C$};
		\node [style=none] (9) at (1.5, 2.5) {$A$};
		\node [style=none] (10) at (0.5, 2.25) {};
		\node [style=none] (11) at (0.5, 1.5) {};
		\node [style=none] (12) at (1.5, 2.25) {};
		\node [style=none] (13) at (1.5, -0.25) {};
		\node [style=none] (14) at (-0.5, 2.25) {};
		\node [style=none] (15) at (-0.5, 2.5) {$B$};
		\node [style=none] (16) at (1.5, -0.5) {$C$};
		\node [style=none] (17) at (0, 1) {};
		\node [style=none] (18) at (1, 1) {};
		\node [style=none] (19) at (1.5, 1.5) {};
	\end{pgfonlayer}
	\begin{pgfonlayer}{edgelayer}
		\draw [in=90, out=-90, looseness=0.75] (2.center) to (3.center);
		\draw (3.center) to (1.center);
		\draw (4.center) to (0.center);
		\draw (10.center) to (11.center);
		\draw [in=90, out=-90, looseness=0.75] (11.center) to (7.center);
		\draw (14.center) to (2.center);
		\draw (7.center) to (13.center);
		\draw [in=180, out=90] (4.center) to (17.center);
		\draw (17.center) to (18.center);
		\draw [in=-90, out=0] (18.center) to (19.center);
		\draw (19.center) to (12.center);
	\end{pgfonlayer}
\end{tikzpicture}
\end{aligned}
\quad
=
\quad
\begin{aligned}
\begin{tikzpicture}
	\begin{pgfonlayer}{nodelayer}
		\node [style=none] (0) at (-0.5, -0.25) {};
		\node [style=none] (1) at (0.5, -0.25) {};
		\node [style=none] (2) at (-0.5, 1) {};
		\node [style=none] (3) at (0.5, 1) {};
		\node [style=none] (4) at (0.5, 0) {};
		\node [style=none] (5) at (-0.5, 0) {};
		\node [style=none] (6) at (-0.5, -0.5) {$A$};
		\node [style=none] (7) at (0.5, -0.5) {$B$};
		\node [style=none] (8) at (1.5, 1) {};
		\node [style=none] (9) at (1.5, 2) {};
		\node [style=none] (10) at (0.5, 2.5) {$C$};
		\node [style=none] (11) at (1.5, 2.5) {$A$};
		\node [style=none] (12) at (0.5, 2.25) {};
		\node [style=none] (13) at (0.5, 1) {};
		\node [style=none] (14) at (0.5, 2) {};
		\node [style=none] (15) at (1.5, 2.25) {};
		\node [style=none] (16) at (1.5, -0.25) {};
		\node [style=none] (17) at (-0.5, 2.25) {};
		\node [style=none] (18) at (-0.5, 2.5) {$B$};
		\node [style=none] (19) at (1.5, -0.5) {$C$};
	\end{pgfonlayer}
	\begin{pgfonlayer}{edgelayer}
		\draw [in=90, out=-90, looseness=0.75] (2.center) to (4.center);
		\draw (4.center) to (1.center);
		\draw [in=90, out=-90, looseness=0.75] (3.center) to (5.center);
		\draw (5.center) to (0.center);
		\draw (12.center) to (14.center);
		\draw [in=90, out=-90, looseness=0.75] (14.center) to (8.center);
		\draw (15.center) to (9.center);
		\draw [in=90, out=-90, looseness=0.75] (9.center) to (13.center);
		\draw (17.center) to (2.center);
		\draw (8.center) to (16.center);
	\end{pgfonlayer}
\end{tikzpicture}
\end{aligned}
\]
Including these identity into our collection of allowable transformations of diagrams gives coherence theorem for symmetric monoidal categories.

\begin{theorem}[Coherence of the graphical calculus for symmetric monoidal categories]
Two morphisms in a symmetric monoidal category are equal with their equality following from the axioms of symmetric monoidal categories if and only if their diagrams are equal up to planar deformation and local applications of the identities \ref{swapdiag1} and \ref{swapdiag2}.
\end{theorem}
\begin{proof}
Joyal-Street \cite[Theorem 2.3]{JS}.
\end{proof}


\section{Hypergraph categories}
\subsection{Frobenius monoids}
\begin{definition}
  A \define{special commutative Frobenius monoid} $(X,\mu,\eta,\delta,\epsilon)$
  in a monoidal category $(\mathcal C, \otimes)$ is an object $X$ of $\mathcal
  C$ together with maps 
\[
  \xymatrixrowsep{1pt}
  \xymatrix{
    \mult{.075\textwidth} & & \unit{.075\textwidth} & & 
    \comult{.075\textwidth} & & \counit{.075\textwidth} \\
    \mu\maps X\otimes X \to X & & \eta\maps I \to X & & 
    \delta\maps X\to X \otimes X & & \epsilon\maps X \to I
  }
\]
obeying the commutative monoid axioms
\[
  \xymatrixrowsep{1pt}
  \xymatrixcolsep{25pt}
  \xymatrix{
    \assocl{.1\textwidth} = \assocr{.1\textwidth} & \unitl{.1\textwidth} =
    \idone{.1\textwidth} & \commute{.1\textwidth} = \mult{.07\textwidth} \\
    \textrm{(associativity)} & \textrm{(unitality)} & \textrm{(commutativity)}
  }
\]
the cocommutative comonoid axioms
\[
  \xymatrixrowsep{1pt}
  \xymatrixcolsep{25pt}
  \xymatrix{
    \coassocl{.1\textwidth} = \coassocr{.1\textwidth} & \counitl{.1\textwidth} =
    \idone{.1\textwidth} & \cocommute{.1\textwidth} = \comult{.07\textwidth} \\
    \textrm{(coassociativity)} & \textrm{(counitality)} &
    \textrm{(cocommutativity)}
  }
\]
and the Frobenius and special axioms
  \[
  \xymatrixrowsep{1pt}
  \xymatrixcolsep{25pt}
  \xymatrix{
    \frobs{.1\textwidth} = \frobx{.1\textwidth} = \frobz{.1\textwidth} & \spec{.1\textwidth} =
    \idone{.1\textwidth} \\
    \textrm{(Frobenius)} & \textrm{(special)} 
  }
  \]
where $\swap{1em}$ is the braiding on $X \otimes X$.   
\end{definition}

In addition to the `upper' unitality law above, the mirror image `lower'
unitality law also holds, due to commutativity and the naturality of the
braiding. While we write two equations for the Frobenius law, this is redundant:
the equality of any two of the expressions implies the equality of all three.
Note that a monoid and comonoid obeying the Frobenius law is commutative if and
only if it is cocommutative.  Thus while a commutative and cocommutative
Frobenius monoid might more properly be called a bicommutative Frobenius monoid,
there is no ambiguity if we only say commutative.

The Frobenius law and the special law go back to Carboni and Walters, under the
names S=X law and the diamond=1 law respectively \cite{CW}. Special commutative
Frobenius monoids is a more modern name; Carboni and Walters termed them
commutative separable algebras \cite{RSW}, 

\begin{definition}
  A \define{hypergraph category} is a symmetric monoidal category in which each
  object $X$ is equipped with a special commutative Frobenius structure
  $(X,\mu_X,\delta_X,\eta_X,\epsilon_X)$ such that 
\[
  \begin{array}{cc}
    \mu_{X\otimes Y} = (\mu_X \otimes \mu_Y)\circ(1_X \otimes \sigma_{YX}\otimes
    1_Y) \qquad&
    \eta_{X\otimes Y} = \eta_X \otimes \eta_Y \\
    \delta_{X\otimes Y} = (1_X \otimes \sigma_{XY}\otimes 1_Y)\circ(\delta_X
    \otimes \delta_Y) \qquad&
    \epsilon_{X\otimes Y} = \epsilon_X \otimes \epsilon_Y.
  \end{array}
\]
\end{definition}
In string diagrams these axioms become trivial:
\[
    \begin{aligned}
\resizebox{.08\textwidth}{!}{
\begin{tikzpicture}
	\begin{pgfonlayer}{nodelayer}
		\node [style=none] (0) at (1, -0) {};
		\node [style=circ] (1) at (0.125, -0) {};
		\node [style=none] (2) at (-1, 0.5) {};
		\node [style=none] (3) at (-1, -0.5) {};
		\node [style=none] (4) at (1, -0.5) {};
		\node [style=none] (5) at (-1, -0) {};
		\node [style=none] (6) at (-1, -1) {};
		\node [style=circ] (7) at (0.125, -0.5) {};
	\end{pgfonlayer}
	\begin{pgfonlayer}{edgelayer}
		\draw [line width=2pt] (0.center) to (1.center);
		\draw [line width=2pt, in=0, out=120, looseness=1.20] (1.center) to (2.center);
		\draw [line width=2pt, in=0, out=-120, looseness=1.20] (1.center) to (3.center);
		\draw [line width=2pt] (4.center) to (7);
		\draw [line width=2pt, in=0, out=120, looseness=1.20] (7) to (5.center);
		\draw [line width=2pt, in=0, out=-120, looseness=1.20] (7) to (6.center);
	\end{pgfonlayer}
      \end{tikzpicture}}
    \end{aligned}
      \quad = \quad
      \begin{aligned}
      \resizebox{.12\textwidth}{!}{
  \begin{tikzpicture}
	\begin{pgfonlayer}{nodelayer}
		\node [style=none] (0) at (1, -0) {};
		\node [style=circ] (1) at (0.125, -0) {};
		\node [style=none] (2) at (-1, 0.5) {};
		\node [style=none] (3) at (-1, -0.5) {};
		\node [style=none] (4) at (1, -1.75) {};
		\node [style=none] (5) at (-1, -1.25) {};
		\node [style=none] (6) at (-1, -2.25) {};
		\node [style=circ] (7) at (0, -1.75) {};
		\node [style=none] (8) at (-2.5, -0.5) {};
		\node [style=none] (9) at (-2.5, -1.25) {};
		\node [style=none] (10) at (-2.5, -2.25) {};
		\node [style=none] (11) at (-2.5, 0.5) {};
	\end{pgfonlayer}
	\begin{pgfonlayer}{edgelayer}
		\draw [line width=2pt] (0.center) to (1.center);
		\draw [line width=2pt, in=0, out=120, looseness=1.20] (1.center) to (2.center);
		\draw [line width=2pt, in=0, out=-120, looseness=1.20] (1.center) to (3.center);
		\draw [line width=2pt] (4.center) to (7);
		\draw [line width=2pt, in=0, out=120, looseness=1.20] (7) to (5.center);
		\draw [line width=2pt, in=0, out=-120, looseness=1.20] (7) to (6.center);
		\draw [line width=2pt] (11.center) to (2.center);
		\draw [line width=2pt, in=180, out=0, looseness=1.00] (8.center) to (5.center);
		\draw [line width=2pt, in=180, out=0, looseness=1.00] (9.center) to (3.center);
		\draw [line width=2pt] (10.center) to (6.center);
	\end{pgfonlayer}
\end{tikzpicture}
}
    \end{aligned}
\]
where the left hand side of the equations are new notation to represent
$\mu_{X\ot Y}$ and $\eta_{X \ot Y}$ respectively. The remaining two axioms are
the mirror image of these.

Note that we do \emph{not} require these Frobenius morphisms to be natural in
$X$.

Note hypergraph structure is not seen by the morphisms in the
category, but by the functors. In that sense it's more 2-categorical.

Related: Let $\mc H \to \mc H'$ be a fully faithful, essentially surjective,
hypergraph functor. Then $\mc H$ and $\mc H'$ are not necessarily equivalent as
hypergraph categories.

\begin{definition}
A functor $(F,\varphi)$ of hypergraph categories, or \define{hypergraph
functor}, is a strong symmetric monoidal functor $(F,\varphi)$ such that for
each object $X$ the following diagrams commute:
\[
  \xymatrix{
    FX\boxtimes FX \ar[rr]^{\mu_{FX}} \ar[dr]_{\varphi} && FX \\
    & F(X \ot X) \ar[ur]_{F\mu_X}
  }
  \qquad
  \xymatrix{
    1_{\mc D} \ar[rr]^{\eta_{FX}} \ar[dr]_{\varphi_1} && FX \\
    & F1_{\mc C} \ar[ur]_{F\eta_X}
  }
\]
\[
  \xymatrix{
    FX \ar[rr]^{\delta_{FX}} \ar[dr]_{F\delta_X} && FX \boxtimes FX\\
    & F(X \ot X) \ar[ur]_{\varphi^{-1}}
  }
  \qquad
  \xymatrix{
    FX \ar[rr]^{\epsilon_{FX}} \ar[dr]_{F\epsilon_X} && 1_{\mc D} \\
    & F1_{\mc C} \ar[ur]_{\varphi^{-1}}
  }
\]
\end{definition}

This means the special commutative Frobenius structure on $FX$ is
\[
  (FX,\enspace F\mu_X \circ \varphi_{X,X},\enspace  \varphi^{-1}_{X,X} \circ F\delta_X,\enspace  F\eta_X \circ
  \varphi_1,\enspace  \varphi_1^{-1} \circ F\epsilon_X).
\]

Just as monoidal natural transformations themselves are enough as morphisms
between symmetric monoidal functors, so too they suffice as morphisms between
hypergraph functors. Two hypergraph categories are \define{hypergraph
equivalent} if there exist hypergraph functors with
monoidal natural transformations to the identity functors. 
Write $\mathrm{HypCat}$ for the (2-)category of hypergraph categories.

\begin{remark}
  This terminology was introduced recently \cite{Ki}, in reference to the fact
that these special commutative Frobenius monoids provide precisely the structure
required to draw graphs with `hyperedges': wires connecting any number of
inputs to any number of outputs. Hypergraph categories have held a number of
names. They were first defined by Walters and Carboni with the name
well-supported compact closed categories \cite{Ca}. In recent years they have
been rediscovered a number of times, also appearing under the names dungeon
categories \cite{Mo} and dgs-monoidal categories. 
\end{remark}


\begin{remark}
  Hypergraph categories are self-dual compact closed. 
Note that if an object $X$ is equipped with a Frobenius monoid structure then
the maps 
\[
    \xymatrixrowsep{0pt}
    \xymatrix{
  \begin{aligned}
      \resizebox{.09\textwidth}{!}{
	\begin{tikzpicture}
	  \begin{pgfonlayer}{nodelayer}
	    \node [style=circ] (0) at (0.75, -0) {};
	    \node [style=circ] (1) at (0.125, -0) {};
	    \node [style=none] (2) at (-1, 0.5) {};
	    \node [style=none] (3) at (-1, -0.5) {};
	  \end{pgfonlayer}
	  \begin{pgfonlayer}{edgelayer}
	    \draw [line width=2pt] (0.center) to (1.center);
	    \draw [line width=2pt, in=0, out=120, looseness=1.20] (1.center) to (2.center);
	    \draw [line width=2pt, in=0, out=-120, looseness=1.20] (1.center) to (3.center);
	  \end{pgfonlayer}
	\end{tikzpicture} 
    }
  \end{aligned}
  & \quad \mbox{and} \quad&
  \begin{aligned}
      \resizebox{.09\textwidth}{!}{
	\begin{tikzpicture}
	  \begin{pgfonlayer}{nodelayer}
	    \node [style=circ] (0) at (-1, 0) {};
	    \node [style=circ] (1) at (-0.375, 0) {};
	    \node [style=none] (2) at (0.75, -0.5) {};
	    \node [style=none] (3) at (0.75, 0.5) {};
	  \end{pgfonlayer}
	  \begin{pgfonlayer}{edgelayer}
	    \draw [line width=2pt] (0.center) to (1.center);
	    \draw [line width=2pt, in=180, out=-60, looseness=1.20] (1.center) to (2.center);
	    \draw [line width=2pt, in=180, out=60, looseness=1.20] (1.center) to (3.center);
	  \end{pgfonlayer}
	\end{tikzpicture}
      } 
  \end{aligned} \\
      \epsilon \circ \mu\maps X \ot X \to 1 & &
      \delta \circ \eta\maps 1 \to X \ot X
    }
\]
obey both
\[
  \begin{aligned}
    \resizebox{3cm}{!}{
      \begin{tikzpicture}
	\begin{pgfonlayer}{nodelayer}
	  \node [style=circ] (0) at (-1.5, 0.5) {};
	  \node [style=circ] (1) at (-0.75, 0.5) {};
	  \node [style=none] (2) at (0.25, -0) {};
	  \node [style=none] (3) at (0.25, 1) {};
	  \node [style=circ] (4) at (1, -0.5) {};
	  \node [style=none] (5) at (0, -0) {};
	  \node [style=circ] (6) at (1.75, -0.5) {};
	  \node [style=none] (7) at (0, -1) {};
	  \node [style=none] (8) at (2.5, 1) {};
	  \node [style=none] (9) at (-2.5, -1) {};
	\end{pgfonlayer}
	\begin{pgfonlayer}{edgelayer}
	  \draw [line width=2pt, in=180, out=-60, looseness=1.20] (1) to (2.center);
	  \draw [line width=2pt, in=180, out=60, looseness=1.20] (1) to (3.center);
	  \draw [line width=2pt] (0.center) to (1);
	  \draw [line width=2pt] (6.center) to (4);
	  \draw [line width=2pt, in=0, out=120, looseness=1.20] (4) to (5.center);
	  \draw [line width=2pt, in=0, out=-120, looseness=1.20] (4) to (7.center);
	  \draw [line width=2pt] (3.center) to (8.center);
	  \draw [line width=2pt] (7.center) to (9.center);
	\end{pgfonlayer}
      \end{tikzpicture}
    }
  \end{aligned}
  \quad = \quad
  \begin{aligned}
    \resizebox{3cm}{!}{
      \begin{tikzpicture}
	\begin{pgfonlayer}{nodelayer}
	  \node [style=circ] (0) at (-0.5, -0) {};
	  \node [style=none] (1) at (-1.5, -0.5) {};
	  \node [style=circ] (2) at (-1.5, 0.5) {};
	  \node [style=circ] (3) at (0.5, -0) {};
	  \node [style=circ] (4) at (1.5, -0.5) {};
	  \node [style=none] (5) at (1.5, 0.5) {};
	  \node [style=none] (6) at (2.5, 0.5) {};
	  \node [style=none] (7) at (-2.5, -0.5) {};
	\end{pgfonlayer}
	\begin{pgfonlayer}{edgelayer}
	  \draw [line width=2pt, in=0, out=-120, looseness=1.20] (0.center) to (1.center);
	  \draw [line width=2pt, in=0, out=120, looseness=1.20] (0.center) to (2.center);
	  \draw [line width=2pt, in=180, out=-60, looseness=1.20] (3) to (4.center);
	  \draw [line width=2pt, in=180, out=60, looseness=1.20] (3) to (5.center);
	  \draw [line width=2pt] (0) to (3);
	  \draw [line width=2pt] (7.center) to (1.center);
	  \draw [line width=2pt] (5.center) to (6.center);
	\end{pgfonlayer}
      \end{tikzpicture}
    }
  \end{aligned}
  \quad = \quad
  \begin{aligned}
    \resizebox{2cm}{!}{
      \begin{tikzpicture}
	\begin{pgfonlayer}{nodelayer}
	  \node [style=none] (0) at (2, -0) {};
	  \node [style=none] (1) at (-2, -0) {};
	  \node [style=none] (2) at (0, -0.5) {};
	  \node [style=none] (3) at (0, 0.5) {};
	\end{pgfonlayer}
	\begin{pgfonlayer}{edgelayer}
	  \draw [line width=2pt](1.center) to (0.center);
	\end{pgfonlayer}
      \end{tikzpicture}
    }
  \end{aligned}
\]
and the reflected equations. Thus if an object carries a Frobenius monoid it is
also self-dual, and any hypergraph category is a fortiori self-dual compact
closed. 

As in any self-dual compact closed category, mapping each morphism $f\colon  X
\to Y$ to its dual morphism
\[
  \big((\epsilon_Y \circ \mu_Y) \otimes 1_X\big) \circ \big( 1_Y \otimes f
  \otimes 1_X \big) \circ \big(1_Y \otimes (\delta_X \circ \eta_X)\big)\colon  Y
  \longrightarrow X
\]
further equips each hypergraph category with a so-called dagger functor---an
involutive contravariant endofunctor that is the identity on objects---such that
the category is a dagger compact category. Dagger compact categories were first
introduced in the context of categorical quantum mechanics \cite{AC}, under the
name strongly compact closed category, and have been demonstrated to be a key
structure in diagrammatic reasoning and the logic of quantum mechanics.
\end{remark}

\section{Coherence}

The objects of a strict monoidal category form a monoid. The monoidal category
is known as objectwise-free if this monoid is isomorphic to a free monoid.

\begin{proposition}
  Every hypergraph category is equivalent as a hypergraph category to a
  objectwise-free strict hypergraph category.
\end{proposition}
\begin{proof}
  Let $(\mc H,\ot)$ be a hypergraph category. As $\mc H$ is a fortiori a
  symmetric monoidal category, Mac Lane's coherence theorem constructs
  an equivalent objectwise-free strict symmetric monoidal category $(\mc
  H_{\mathrm{str}}, \cdot)$ with objects finite lists $[x_1,\dots,x_n]$ of
  objects of $\mc H$ and morphisms $[x_1,\dots,x_n] \to [y_1,\dots,y_m]$ those
  morphisms from $((x_1 \ot x_2) \ot \dots) \ot x_n \to ((y_1 \ot y_2) \ot
  \dots) \ot y_m$ in $\mc H$.  Composition is given by composition in $\mc H$.
  
  The monoidal structure is given as follows. Given a list $X$ of objects in
  $\mc H$, write $PX$ for the corresponding monoidal product in $\mc H$ with all
  open parathesis at the front.  The monoidal product of two objects is given by
  concatenation $\cdot$ of lists; the monoidal unit is the empty list. The
  monoidal product of two morphisms is given by their monoidal product in $\mc
  H$ pre- and post-composed with the necessary canonical maps: given $f\maps X
  \to Y$ and $g\maps Z \to W$, their product $f\cdot g\maps X\cdot Y \to Z \cdot
  W$ is 
  \[
    P(X \cdot Y) \longrightarrow PX \ot PY \stackrel{f \ot g}{\longrightarrow}
    PZ \ot PW \longrightarrow P(Z \cdot W).
  \]
  By design, the associators and unitors are simply identity maps. The braiding
  $X \cdot Y \to Y \cdot X$ is given by the braiding $PX \ot PY \to PY \ot PX$
  in $\mc H$, similarly pre- and post-composed with the necessary canonical
  maps. This defines a strict symmetric monoidal category \cite{Mac98}.

  To make $\mc H_{\mathrm{str}}$ into a hypergraph category, we equip each
  object $X$ with a special commutative Frobenius monoid in the same way: we
  take the special commutative Frobenius monoid on $PX$ and compose with the
  necessary canonical maps to get morphisms of the desired type. For example,
  the multiplication on $X$ is given by 
  \[
    P(X \cdot X) \longrightarrow PX \ot PX \stackrel{\mu_X}\longrightarrow PX.
  \]
  This hypergraph structure is well-defined. To check this, observe that the
  hypergraph structure on $[x_1,\dots,x_n]\cdot[y_1,\dots,y_m]$ is given by 

  We thus have a hypergraph category.
  \begin{center}
    \begin{tabular}{| c | p{.65\textwidth} |}
      \hline
      \multicolumn{2}{|c|}{The strict hypergraph category $(\mc H_{\mathrm{str}},
      \cdot)$} \\
      \hline
      \textbf{objects} & finite lists $[x_1, \dots, x_n]$ of objects of
      $\mathcal H$ \\ 
      \textbf{morphisms} & $\mathrm{hom}_{\mc H_{\mathrm{str}}}\big([x_1, \dots,
      x_n],[y_1, \dots, y_m]\big)$ \newline $= \mathrm{hom}_{\mc H}\big(((x_1 \ot x_2) \ot
      \dots) \ot x_n, ((y_1 \ot y_2) \ot \dots) \ot y_m\big)$\\ 
      \textbf{composition} & composition of corresponding maps in $\mc H$ \\
      \textbf{monoidal product} & concatenation of lists on objects, on
      morphisms inherited from $\mc H$ \\
      \textbf{coherence maps} & associators and unitors are strict; braiding is
      inherited from $\mc H$\\
      \textbf{hypergraph maps} & inherited from $\mc H$ \\
      \hline
    \end{tabular}
  \end{center}

  Mac Lane's equivalence is witnessed by strong symmetric monoidal functors
  $P\maps \mc H_{\mathrm{str}} \to \mc H$ extending the map $P$ above, and
  $S\maps \mc H \to \mc H_{\mathrm{str}}$ sending $x \in \mc H$ to the string
  $[x]$ of length 1 in $\mc H_{\mathrm{str}}$. These extend to hypergraph
  functors.

  In detail, the functor $P$ is given on morphisms by taking a map in
  $\mathrm{hom}_{\mc H_{\mathrm{str}}}(X,Y)$ to the same map considered now as a
  map in $\mathrm{hom}_{\mc H}(PX,PY)$; its coherence maps are given by the
  canonical maps $PX \ot PY \to P(X\cdot Y)$. The functor $S$ is even easier to
  define: a morphism $x \to y$ in $\mc H$ is by definition a morphism $[x] \to
  [y]$ in $\mc H_{\mathrm{str}}$, so $S$ is a monoidal embedding of $\mc H$ into
  $\mc H_{\mathrm{str}}$. 
  
  By Mac Lane's proof of the coherence theorem for monoidal categories these are
  both strong monoidal functors; by inspection they also preserve hypergraph
  structure, and so are hypergraph functors.  As they already witness an
  equivalence of symmetric monoidal categories, thus $\mc H$ and $\mc
  H_{\mathrm{str}}$ are equivalent as hypergraph categories.
\end{proof}


Note that two morphisms are equal if their string diagrams are equal via
Frobenius laws, symmetry laws, and topological deformation.

\begin{conjecture}
  Graphical calculus for hypergraph categories: two morphisms are equal if and
  only if their string diagrams are equivalent via Frobenius laws and
  topological deformation.
\end{conjecture}

One might prove this using coherence theorem and spider theorem.


\section{Example: cospan categories}

  A central example of a hypergraph category is the category
  $\mathrm{Cospan(\mathcal C)}$ of cospans in any category $\mathcal C$ with
  finite colimits. We will later see that decorated cospan categories are a
  generalisation of such categories, and each inherits a hypergraph structure
  from such. 

  Let $\mc C$ be a category with finite colimits.
Recall that a \define{cospan} $X \stackrel{i}{\longrightarrow} N
\stackrel{o}{\longleftarrow} Y$  from $X$ to $Y$ in $\mathcal C$ is a pair of
morphisms with common codomain. We refer to $X$ and $Y$ as the \define{feet},
and $N$ as the \define{apex}.  Given two cospans $X
\stackrel{i}{\longrightarrow} N \stackrel{o}{\longleftarrow} Y$ and $X
\stackrel{i'}{\longrightarrow} N' \stackrel{o'}{\longleftarrow} Y$ with the same
feet, a \define{map of cospans} is a morphism $n\colon  N \to N'$ in $\mathcal
C$ between the apices such that
\[
  \xymatrix{
    & N \ar[dd]^n  \\
    X \ar[ur]^{i} \ar[dr]_{i'} && Y \ar[ul]_{o} \ar[dl]^{o'}\\
    & N'
  }
\]
commutes.

Cospans may be composed, up to isomorphism, using the pushout from the common
foot: given cospans $X \stackrel{i_X}{\longrightarrow} N
\stackrel{o_Y}{\longleftarrow} Y$ and $Y \stackrel{i_Y}{\longrightarrow} M
\stackrel{o_Z}{\longleftarrow} Z$, their composite cospan is $X \stackrel{j_N
  \circ i_X}{\longrightarrow} N+_YM \stackrel{j_M\circ i_Z}{\longleftarrow} Z$,
  where 
\[
  \xymatrix{
    && N+_YM \\
    & N \ar[ur]^{j_N} && M \ar[ul]_{j_M} \\
    \quad X \quad \ar[ur]^{i_X} && Y \ar[ul]_{o_Y} \ar[ur]^{i_Y} && \quad Z \quad \ar[ul]_{o_Z}
  }
\]
is a pushout square. 

We consider any category $\mathcal C$ as a symmetric monoidal category with
monoidal product given by the coproduct, written $+$, and braiding given by the
maps $A+B \to B+A$ by copairing identity maps. 

Given maps $f \maps A \to C$, $g \maps B \to C$ with common codomain, the
universal property of the coproduct gives a unique map $\maps A+B \to C$. We
call this the \define{copairing} of $f$ and $g$, and write it $[f,g]$. 

Given a category $\mathcal C$ with pushouts, we may define a symmetric monoidal
category $\mathrm{Cospan}(\mathcal C)$ with objects the objects of $\mathcal C$
and morphisms isomorphism classes of cospans \cite{Be}. 

  First, $\mathrm{Cospan(\mathcal C)}$ inherits a symmetric monoidal structure
  from $\mathcal C$. We call a subcategory $\mathcal C$ of a category $\mathcal
  D$ \define{wide} if $\mathcal C$ contains all objects of $\mathcal D$, and
  call a functor that is faithful and bijective-on-objects a \define{wide
  embedding}. Note then that we have a wide embedding
  \[
    \mathcal C \hooklongrightarrow \mathrm{Cospan(\mathcal C)}
  \]
  that takes each object of $\mathcal C$ to itself as an object of
  $\mathrm{Cospan(\mathcal C)}$, and each morphism $f\colon  X \to Y$ in $\mathcal C$
  to the cospan
  \[
    \xymatrix{
      & Y \\
      X \ar[ur]^{f} && Y, \ar@{=}[ul]
    }
  \]
  where the extended `equals' sign denotes an identity morphism. This allows us
  to view $\mathcal C$ as a wide subcategory of $\mathrm{Cospan(\mathcal C)}$.

  Now as $\mathcal C$ has finite colimits, it can be given a symmetric monoidal
  structure with the coproduct the monoidal product; we write this monoidal
  category $(\mathcal C,+)$, and write $\varnothing$ for the initial object, the
  monoidal unit of this category. Then $\mathrm{Cospan(\mathcal C)}$ inherits
  the same symmetric monoidal structure: since the monoidal product $+\colon \mathcal
  C \times \mathcal C \to \mathcal C$ is left adjoint to the diagram functor, it
  preserves colimits, and so extends to a functor $+\colon
  \mathrm{Cospan(\mathcal C)} \times \mathrm{Cospan(\mathcal C)} \to
  \mathrm{Cospan(\mathcal C)}$. The remainder of the monoidal structure is
  inherited because $\mathcal C$ is a wide subcategory of
  $\mathrm{Cospan(\mathcal C)}$.

  Next, the Frobenius structure comes from copairings of identity morphisms. We
  call cospans 
  \[
    \xymatrix{
      & N \\
      X \ar[ur]^{i} && Y \ar[ul]_{o}
    }
    \qquad \xymatrix@R=8pt{\\\textrm{and}} \qquad 
    \xymatrix{
      & N \\
      Y \ar[ur]^{o} && X \ar[ul]_{i}
    }
  \]
  that are reflections of each other \define{opposite} cospans. Given any object
  $X$ in $\mathcal C$, the copairing $[1_X,1_X]\colon  X + X \to X$ of two identity
  maps on $X$, together with the unique map $!\colon  \varnothing \to X$ from the
  initial object to $X$, define a monoid structure on $X$. Considering these
  maps as morphisms in $\mathrm{Cospan(\mathcal C)}$, we may take them together
  with their opposites to give a special commutative Frobenius structure on $X$.
  In this way we consider each category $\mathrm{Cospan(\mathcal C)}$ a
  hypergraph category.

\begin{definition} \label{thm.cospanwelldef}
  Let $\mc C$ be a category with finite colimits. We define the hypergraph
  category $\mathrm{Cospan}(\mathcal C)$ to comprise:
  
  \begin{center}
  \begin{tabular}{ |c| p{.65\textwidth}|}
      \hline
      \multicolumn{2}{|c|}{The hypergraph category $(\mathrm{Cospan(\mc C)},+)$} \\
    \hline
    \textbf{objects} & the objects of $\mathcal C$ \\ 
    \textbf{morphisms} & isomorphism classes of cospans in
    $\mathcal C$\\ 
  \textbf{composition} & given by pushout \\
  \textbf{tensor product} & the coproduct in $\mathcal C$. \\
  \textbf{coherence maps} & inherited from $(\mc C,+)$ $\sigma_{X,Y} = [\iota_Y,\iota_X] \maps X+Y
      \to Y+X$\\
  \textbf{hypergraph maps} & $\mu_X = [1_X,1_X]$, $\eta_X = !$,
      $\delta_X = \mu_X^\opp$, $\epsilon_X = \eta_X^\opp$. \\
      \hline
  \end{tabular}
\end{center}
\end{definition}
\begin{notation}
  We will often abuse our
terminology and refer to cospans themselves as morphisms in some cospan
category $\mathrm{Cospan}(\mathcal C)$; we of course refer instead to the
isomorphism class of the said cospan.


  Given $f \maps X \to Y$ in $\mc C$, we also abuse notation by writing $f \in
  \mathrm{Cospan}(\mc C)$ for the cospan $X \stackrel{f}\to Y
  \stackrel{1_Y}\leftarrow Y$, and $f^\opp$ for the cospan $Y \stackrel{1_Y}\to
  Y \stackrel{f}\leftarrow X$.
\end{notation}

%\begin{proposition}
%  The monoidal product in a hypergraph category is a coproduct for something if
%  and only if every morphism is a Frobenius monoid homomorphism. (ie take
%  subcategory of all objects, monoid maps, and monoid homs. This is cocomplete.)
%\end{proposition}
%Clearly not true for circuits for example.



  
FinCocompleteCat faithfully embeds into HyperCat. ie any monoidal category of
cospans has a hypergraph structure inherited from the identity morphisms.

  Hypergraph categories are closely related to cospans. The free hypergraph
  category on a single object in the category of cospans in the category of
  finite sets. SpivakVagner?

Walters: cospan graph is the generic special commutative Frobenius monoid.

Later, also Vagner Spivak Schultz: hypergraph categories are algebras of cospan.

