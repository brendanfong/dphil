\chapter{Decorated corelations}

When enough structure is available to us, we may decorate corelations too.
Furthermore, and key to the idea of `black-boxing', we get a hypergraph functor
from decorated cospans to decorated corelations.

\section{Introduction}

Semantics live on boundaries only.

We then introduce a new framework for working with hypergraph
categories: decorated corelations.

Decorated corelations adds compositional operations to network-diagram
representations, and handles composition of semantics too. 

Two main theorems: 
\begin{theorem}
Given a category $\mc C$ with finite colimits, factorisation system $(\mc E,\mc
M)$ such that $\mc M$ is stable under pushouts, and a symmetric lax monoidal functor 
\[
  \mc C; \mc M^\opp \longrightarrow \Set,
\]
we may define a hypergraph category of with morphisms decorated corelations.
\end{theorem}

\begin{theorem}
Every hypergraph category can be constructed in this way.
\end{theorem}

They apply to functors too.

\section{Decorated corelations} \label{sec:dcorc}

The key difference is to decorate cospans we need to know how to push
decorations up. To decorate corelations we all need to know how to pull
decorations back down. This is related to the existence of an extraspecial
commutative Frobenius monoid in our main applications.



\subsection{Adjoining right adjoints}

Suppose we have a cospan $X+Y \to N$ with a decoration on $N$. Reducing this to
a corelation requires us to factor this to $X+Y \stackrel{e}\to \overline{N}
\stackrel{m}\to N$. To define a category of decorated corelations, then, we must
specify how to take decoration on $N$ and `pull it back' along $m$ to a decoration on
$\overline{N}$.

For decorated cospans, it is enough to have a functor $F$ from a category $\mc C$
with finite colimits; the image $Ff$ of morphisms $f$ in $\mc C$ describes how
to move decorations forward along $f$. In this subsection we explain how to
expand $\mc C$ to include morphisms $m^\opp$ for each $m \mc C$, so that the
image of $m^\opp$ describes how to move morphisms backwards along $m$.

\begin{proposition}
  Let $\mathcal C$ be a category with finite colimits, and let $\mathcal M$ be a
  subcategory of $\mathcal C$ stable under pushouts. Then we define the category
  $\mathcal C; \mathcal M^\opp$ as follows  
  \begin{center}
    \begin{tabular}{| c | p{.65\textwidth} |}
      \hline
      \multicolumn{2}{|c|}{The symmetric monoidal category $(\mc C;\mc M^\opp,+)$} \\
      \hline
      \textbf{objects} & the objects of $\mathcal C$ \\ 
      \textbf{morphisms} & isomorphism classes of cospans of the form
      $\stackrel{c}\rightarrow \stackrel{m}\leftarrow$, where $c$ lies in
      $\mathcal C$ and $m$ in $\mathcal M$\\ 
      \textbf{composition} & given by pushout \\
      \textbf{monoidal product} & the coproduct in $\mathcal C$ \\
      \textbf{coherence maps} & the coherence maps in $\mc C$ \\
      \hline
    \end{tabular}
  \end{center}
\end{proposition}
\begin{proof}
  This is a symmetric monoidal subcategory of $\cospan(\mc C)$. Our data is
  well-defined: composition because $\mc M$ is stable under pushouts, and
  monoidal composition by Lemma \ref{lem.mcoproductsmc}. 
\end{proof}

This category can be viewed as a bicategory, with 2-morphisms given by maps of
cospans. In this bicategory every morphism of $\mc M$ has a right adjoint.

\begin{examples} 
  Our familiar examples:
  \begin{itemize}
    \item $\mathcal C; \mathcal C^\opp$ is by definition equal to
$\cospan(\mathcal C)$.
\item $\mathcal C;\mathcal I_{\mathcal C}^\opp$ is
naturally isomorphic to $\mathcal C$.
\item $\Set;\mathrm{Inj}^\opp$ is the category of partial functions.
\end{itemize}
\end{examples}

\begin{lemma} \label{lem.madjointsfunctor}
  Let $\mathcal C$, $\mathcal C'$ be categories with finite colimits, and let
  $\mathcal M$, $\mathcal M'$ be subcategories each stable under pushouts. Let
  $A\maps \mathcal C \to \mc C'$ be functor that preserves colimits and such
  that the image of $\mc M$ lies in $\mc M'$. Then $A$ extends to a symmetric
  strong monoidal functor
  \[
    A\maps \mc C;\mc M^\opp \longrightarrow \mc C'; \mc M'^\opp.
  \]
  mapping $X$ to $AX$ and $\stackrel{c}\rightarrow \stackrel{m}\leftarrow$ to
  $\stackrel{Ac}\rightarrow \stackrel{Am}\leftarrow$.
\end{lemma}
\begin{proof}
  Note $A(\mc M) \subseteq \mc M'$, so $\stackrel{Ac}\rightarrow
  \stackrel{Am}\leftarrow$ is indeed a morphism in $\mc C';\mc M'^\opp$. This is
  then a restriction and corestriction of the usual functor $\cospan(\mc C) \to
  \cospan(\mc C')$ to the above domain and codomain.
\end{proof}

That this `subcospan category' construction could be defined more
generally using any two isomorphism-containing wide subcategories stable under
pushout, but the above suffices for decorated corelations. 

\subsection{Decorated corelations}
Decorated corelations are constructed from a lax monoidal functor from $\mc
C;\mc M^\opp$ to $\Set$.

\begin{definition}
  Let $\mathcal C$ be a category with finite colimits, and let $(\mathcal E,
  \mathcal M)$ be a factorisation system on $\mathcal C$. Suppose we also
  have a lax monoidal functor
  \[
    F: (\mathcal C;\mathcal M^\opp,+) \longrightarrow (\Set, \times).
  \]
  We define an $F$-\define{decorated corelation} to a pair
  \[
    \left(
    \begin{aligned}
      \xymatrix{
	& N \\  
	X \ar[ur]^{i} && Y \ar[ul]_{o}
      }
    \end{aligned}
    ,
    \qquad
    \begin{aligned}
      \xymatrix{
	FN \\
	1 \ar[u]_{s}
      }
    \end{aligned}
    \right)
  \]
  where the cospan is jointly $\mathcal E$-like. A morphism of decorated
  corelations is a morphism of decorated cospans between two decorated
  corelations.
\end{definition}

Suppose we have decorated corelations
\[
  \left(
  \begin{aligned}
    \xymatrix{
      & N \\  
      X \ar[ur]^{i} && Y \ar[ul]_{o}
    }
  \end{aligned}
  ,
  \qquad
  \begin{aligned}
    \xymatrix{
      FN \\
      1 \ar[u]_{s}
    }
  \end{aligned}
  \right)
  \qquad
  \mbox{and}
  \qquad
  \left(
  \begin{aligned}
    \xymatrix{
      & M \\  
      Y \ar[ur]^{i} && Z \ar[ul]_{o}
    }
  \end{aligned}
  ,
  \qquad
  \begin{aligned}
    \xymatrix{
      FM \\
      1 \ar[u]_{s}
    }
  \end{aligned}
  \right).
\]
Then their composite is given by the composite corelation
\[
  \xymatrix{
    & \overline{N+_YM} \\  
    X \ar[ur]^{i} && Z \ar[ul]_{o}
  }
\]
paired with the decoration
\[
  1 \longrightarrow F(N+M) \longrightarrow F(N+_YM) \stackrel{F(m^\opp)}\longrightarrow F(\overline{N+_YM})
\]
As composition of corelations and decorated cospans are both well-defined up to
isomorphism, this too is well-defined up to isomorphism. Again, we will be lazy
about the distinction between a decorated corelation and its isomorphism class.


\subsection{Categories of decorated corelations}
In this subsection we give a definition of the hypergraph category
$F\mathrm{Corel}$ of decorated corelations.  In analogy with how we defined the
hypergraph category of corelations, we leverage the fact that decorated cospans
form a hypergraph category, this time using a structure preserving map 
\[
  \square\maps F\mathrm{Cospan} \longrightarrow F\mathrm{Corel}.
\]
Here $F\mathrm{Cospan}$ denotes the decorated cospan category constructed from
the restriction of the functor $F\maps \mc C;\mc M^\opp \to \Set$ to the domain
$\mc C$. 

Given a cospan $X \to N \leftarrow Y$, write $m\maps \overline{N} \to N$ for the
$\mc M$ factor of the copairing $X+Y \to N$. The functor $\square$ takes a
decorated cospan 
\[
  (X \stackrel{i}\longrightarrow N \stackrel{o}\longleftarrow Y, \enspace 1
    \stackrel{s}\longrightarrow FN)
\]
to the decorated corelation 
\[
  (X \stackrel{\overline{i}}\longrightarrow \overline N
  \stackrel{\overline{o}}{\longleftarrow} Y, \enspace 1 \xrightarrow{Fm^\opp \circ
  s} F\overline{N}),
\]
where the corelation is given by the jointly $\mc E$-part of the cospan, and the
decoration is given by composing $s$ with the $F$-image $Fm^\opp\maps FN \to
F\overline{N}$ of the map $N \stackrel{1_N}\to N \stackrel{m}\leftarrow
\overline{N}$ in $\mc C;\mc M^\opp$. We call $Fm^\opp \circ s$ the
\define{restricted decoration} of the decorated cospan $(X \to N \leftarrow Y,
\enspace 1 \stackrel{s}\to FN)$.

The monoidal product of two decorated corelations is their monoidal product as
decorated cospans.

\begin{theorem} \label{thm.fcorel}
  Let $\mathcal C$ be a category with finite colimits and factorisation system
  $(\mathcal E, \mathcal M)$ with $\mathcal M$ stable under pushout, and let 
  \[
    F: (\mathcal C;\mathcal M^\opp,+) \longrightarrow (\Set, \times)
  \]
  be a symmetric lax monoidal functor.  Then we may define 
  \begin{center}
    \begin{tabular}{| c | p{.65\textwidth} |}
      \hline
      \multicolumn{2}{|c|}{The hypergraph category $(F\mathrm{Corel},+)$} \\
      \hline
      \textbf{objects} & the objects of $\mathcal C$ \\ 
      \textbf{morphisms} & isomorphism classes of $F$-decorated corelations in
      $\mathcal C$\\ 
      \textbf{composition} & given by $\mc E$-part of pushout with restricted
      decoration  \\
      \textbf{monoidal product} & the coproduct in $\mathcal C$  \\
      \textbf{coherence maps} & maps from $\cospan(\mc C)$ with restricted empty
      decoration \\
      \textbf{hypergraph maps} & maps from $\cospan(\mc C)$ with restricted empty
      decoration \\
      \hline
    \end{tabular}
  \end{center}
\end{theorem}
Similar to the corelations theorem (Theorem \ref{thm.cospantocorel}), we have
specified well-defined data and now need to check a collection of coherence
axioms. As before, we prove this alongside the theorem about decorated
corelation functors in the next section.

\begin{example}
  Note that decorated cospans are a special case of decorated corelations:
  simply use an morphism--isomorphism factorisation system.
\end{example}

\begin{example} \label{ex.undeccorel}
  Note that `undecorated' corelations are a special case of decorated
  corelations: they are corelations decorated by the functor $1\maps \mc C;\mc
  M^\opp \to \Set$ that maps each object to the one element set $1$, and each
  morphism to the identity function on $1$. This is a symmetric monoidal functor
  with the coherence maps all also the identity function on $1$.
\end{example}


%Associativity: To take a decoration on $A+B$ to one on $A+_C\overline B$ we may
%either reduce to the $\mathcal E$-part of $B$ and then pushout over $C$, or
%pushout over $C$ and then reduce to the $\mathcal E$ part of $B$. This lemma
%implies that both processes result in the same decoration.  i

\section{Functors between decorated corelation categories}
Functors between decorated corelation categories hold no surprises: their
requirements combine the requirements of corelations and decorated cospans.
Recall that Lemma \ref{lem.madjointsfunctor} says that we can extend a
colimit-preserving functor $\mc C \to \mc C'$ to a symmetric monoidal functor
$\mc C;\mc M^\opp \to \mc C';\mc M'^\opp$.

\begin{proposition}\label{prop.deccorelfunctors}
  Let $\mathcal C$, $\mathcal C'$ have finite colimits and respective
factorisation systems $(\mathcal E, \mathcal M)$, $(\mathcal E', \mathcal M')$,
such that $\mathcal M$ and $\mathcal M'$ are stable under pushout, and suppose
that we have symmetric lax monoidal functors
\[
  F: (\mathcal C;\mathcal M^\opp,+) \longrightarrow (\Set, \times)
\]
and
\[
  G: (\mathcal C';\mathcal M'^\opp,+) \longrightarrow (\Set, \times).
\]

Further let $A\maps \mathcal C \to \mathcal C'$ be a functor that preserves
finite colimits and such that the image of $\mathcal M$ lies in $\mathcal M'$.
This functor $A$ extends to a symmetric monoidal functor $\mc C;\mc M^\opp \to
\mc C';\mc M'^\opp$.

Suppose we have a monoidal natural transformation $\theta$:
\[
  \xymatrixrowsep{2ex}
  \xymatrix{
    \mc C; \mc M^\opp \ar[dd]_{A} \ar[drr]^F  \\
    &\twocell \omit{_\:\theta}& \Set \\
    \mc C'; \mc {M'}^\opp \ar[urr]_{G} 
  }
\]

Then we may define a hypergraph functor $T\maps F\mathrm{Corel} \to
G\mathrm{Corel}$ sending each object $X \in F\mathrm{Corel}$ to $AX \in
G\mathrm{Corel}$ and each decorated corelation 
\[
  (X \stackrel{i_X}{\longrightarrow} N \stackrel{o_Y}{\longleftarrow} Y, \quad
  1 \stackrel{s}{\to} FN)
\]
to
\[
  (AX \stackrel{Ai_X}{\longrightarrow} \overline{AN} \stackrel{Ao_Y}{\longleftarrow} AY,
  \quad 1 \stackrel{s}{\to} FN \stackrel{\theta_N}{\to} GAN
  \xrightarrow{Gm_{AN}^\opp} G\overline{AN}).
\]
The coherence maps $\overline{\kappa_{X,Y}})$ are given by the coherence maps of $A$ with the restricted empty decoration.
\end{proposition}
\begin{proof}[Proof of Theorem \ref{thm.fcorel} and Proposition
  \ref{prop.deccorelfunctors}]
  In the proof of Theorem \ref{thm.cospantocorel} and Proposition
  \ref{prop.corelfunctors} we proved that the map 
  \[
    \square\maps \corel(\mc C) \longrightarrow \corel(\mc C')
  \]
  preserved composition and had natural coherence maps. Specialising to the case
  when $\corel(\mc C)=\cospan(\mc C')$, we saw that this bijective-on-objects,
  surjective-on-morphisms, composition and monoidal product preserving map
  proved $\corel(\mc C')$ is a hypergraph category, and it immediately followed
  that $\square$ is a hypergraph functor.

  The analogous argument holds here: we simply need to prove
  \[
    \square\maps F\corel \longrightarrow G\corel
  \]
  preserves composition and has natural coherence maps. Theorem \ref{thm.fcorel}
  then follows from examining the map $F\cospan \to F\corel$ obtained by
  choosing $\mc C = \mc C'$, $(\mc E,\mc M) = (\mc C', \mc I_{\mc C'})$, $F$ the
  restriction of $G$ to $\mc C'$, $A$ the identity functor on $\mc C'$, and
  $\theta$ the identity natural transformation. Subsequently Proposition
  \ref{prop.deccorelfunctors} follows from noting that all the axioms hold for
  the corresponding maps in $G\mathrm{Cospan}$.
  
  \paragraph{$\square$ preserves composition.} Suppose we have decorated corelations
  \[
    f=(X \stackrel{i_X}{\longrightarrow} N \stackrel{o_Y}{\longleftarrow} Y,
    \enspace 1 \stackrel{s}{\to} FN)
    \qquad
    \mbox{and}
    \qquad 
    g=(Y \stackrel{i_Y}{\longrightarrow} M \stackrel{o_Y}{\longleftarrow} Z,
    \enspace 1 \stackrel{t}{\to} FM)
  \]
%  \[
%    \left(
%    \begin{aligned}
%      \xymatrix{
%	& N \\  
%	X \ar[ur]^{i} && Y \ar[ul]_{o}
%      }
%    \end{aligned}
%    ,
%    \qquad
%    \begin{aligned}
%      \xymatrix{
%	FN \\
%	1 \ar[u]_{s}
%      }
%    \end{aligned}
%    \right)
%    \qquad
%    \mbox{and}
%    \qquad
%    \left(
%    \begin{aligned}
%      \xymatrix{
%	& M \\  
%	Y \ar[ur]^{i} && Z \ar[ul]_{o}
%      }
%    \end{aligned}
%    ,
%    \qquad
%    \begin{aligned}
%      \xymatrix{
%	FM \\
%	1 \ar[u]_{s}
%      }
%    \end{aligned}
%    \right).
%  \]
  We know the functor $\square$ preserves composition on the cospan part; this
  is precisely the content of Proposition \ref{prop.corelfunctors}. It remains to
  check that $\square( g \circ f)$ and $\square g \circ \square f$ have
  isomorphic decorations. This is expressed by the commutativity of the
  following diagram:
  \[
    \xymatrixrowsep{1.1pc}
    \xymatrixcolsep{0pc}
    \xymatrix{
      G\overline{A(\overline{N+_YM})} \ar[rrrrrr]^{Gn} &&&&&&
      G\overline{(\overline{AN}+_{AY}\overline{AM})}\\
      \\
      GA(\overline{N+_YM}) \ar[uu]^{Gm^\opp_{A(\overline{N+_YM})}} &&&
      \textsc{\tiny($\ast\ast$)} &&& 
      G(\overline{AN}+_{AY}\overline{AM}) \ar[uu]_{Gm^\opp_{\overline{AN}+_{AY}\overline{AM}}} \\
      \\
      F(\overline{N+_YM}) \ar[uu]^{\theta_{\overline{N+_YM}}} & %\textsc{\tiny()} & 
      &
      GA(N+_YM) \ar[uull]_{GAm^\opp_{N+_YM}} && 
      G(AN+_{AY}AM) \ar[ll]_{G\sim} \ar[uurr]^{G(m_{AN}^\opp +_{AY}m_{AM}^\opp)}
      & \textsc{\tiny(c)} & 
      G(\overline{AN}+\overline{AM}) \ar[uu]_{G[j_{\overline{AN}},j_{\overline{AM}}]} 
      \\
      &\textsc{\tiny(tn)}&& \textsc{\tiny(a)} 
      \\
      F(N+_YM) \ar[uu]^{Fm_{N+_YM}^\opp} %\ar[uurr]_{\theta_{N+_YM}} & \textsc{\tiny()} & 
      &&
      GA(N+M)\ar[uu]_{GA[j_N,j_M]} && 
      G(AN+AM) \ar[uu]^{G[j_{AN},j_{AM}]} \ar[ll]^{G\alpha_{N,M}}
      \ar[uurr]^{G(m_{AN}^\opp +m_{AM}^\opp)} & \textsc{\tiny(gm)} & 
      G\overline{AN} \times G\overline{AM}
      \ar[uu]_{\gamma_{\overline{AN},\overline{AM}}} \\
      \\
      F(N+M) \ar[uu]^{F[j_N,j_M]} \ar[uurr]_{\theta_{N+M}} &&& \textsc{\tiny(tm)} &&& 
      GAN \times GAM \ar[uu]_{Gm_{AN}^\opp \times Gm_{AM}^\opp}
      \ar[uull]^{\gamma_{AN,AM}} \\
      \\
      &&&FN \times FM \ar[uulll]^{\varphi_{N,M}} \ar[uurrr]_{\theta_N \times
      \theta_M} \\\\
      &&&1 \ar[uu]_{\rho_1\circ (s \times t)}
    }
  \]
  This diagram does indeed commute. To check this, first observe that \textsc{(tm)}
  commutes by the monoidality of $\theta$, \textsc{(gm)} commutes by the
  monoidality of $G$, and \textsc{(tn)} commutes by the naturality of $\theta$.
  The remaining three diagrams commute as they are $G$-images of diagrams that
  commute in $\mc C';\mc M'^\opp$. Indeed, \textsc{(a)} commutes since $A$ preserves
  colimits and $G$ is functorial, \textsc{(c)} commutes as it is the $G$-image
  of a pushout square in $\mc C'$, so 
  \[
    \stackrel{m_{AN}^\opp+m_{AM}^\opp}{\longleftarrow} 
    \stackrel{[j_{\overline{AN}},j_{\overline{AM}}]}{\longrightarrow}
    \quad 
    \textrm{and}
    \quad
    \stackrel{[j_{AN},j_{AM}]}{\longrightarrow}
    \stackrel{m_{AN}^\opp+_{AY}m_{AM}^\opp}{\longleftarrow} 
  \]
  are equal as morphisms of $\mc C';\mc M'^\opp$, and \textsc{($\ast\ast$)}
  commutes as it is the $G$-image of the right-hand subdiagram of
  (\ref{diag.eparts}) used to define $n$ in the proof of Lemma
  \ref{lem.corelfuncomposition}.

  \paragraph{Coherence maps are natural.}
  Let $f = (X \longrightarrow N \longleftarrow Y, \enspace 1 \to FN)$, $g= (Z
  \longrightarrow M \longleftarrow W, \enspace 1 \to FM)$ be $F$-decorated
  corelations in $\mc C$. We wish to show that
  \[
    \xymatrixcolsep{4pc}
    \xymatrixrowsep{2pc}
    \xymatrix{
      AX+AY \ar[r]^{\square f+\square g}
      \ar[d]_{\overline{\kappa_{X,Y}}} & 
      AZ+AW \ar[d]^{\overline{\kappa_{Z,W}}} \\
      A(X+Y) \ar[r]^{\square(f+g)} & A(Z+W)
    }
  \]
  commutes in $G\mathrm{Corel}$, where the coherence maps are given by
  \[
    \overline{\kappa_{X,Y}}=          
    \left(
    \begin{aligned}
      \xymatrix{
	& \overline{A(X+Y)} \\  
	AX+AY \ar[ur] && A(X+Y) \ar[ul]
      }
    \end{aligned}
    ,
    \qquad
    \begin{aligned}
      \xymatrixrowsep{1.4ex}
      \xymatrix{
	G(\overline{A(X+Y)}) \\
	GA(X+Y) \ar[u]_{Gm_{AX+AY}^\opp} \\
	G\varnothing \ar[u]_{G!} \\
	1 \ar[u]_{\gamma_1}
      }
    \end{aligned}
    \right).
  \]
  Lemma \ref{lem.corelfunmonoidal} shows that the composites of corelations
  agree. It remains to check that the decorations also agree.

  Here Lemma \ref{lem.emptydecorations} is helpful. Since $\square$ is
  composition preserving, we can replace the $\overline{\kappa}$ with the empty
  decorated coherence maps $\kappa$ of $G\mathrm{Cospan}$, and compute these
  composites in $G\mathrm{Cospan}$, before restricting to the $\mc E'$-parts.
  Lemma \ref{lem.emptydecorations} then implies that the restricted empty
  decorations on the isomorphisms $\overline{\kappa}$ play no role in
  determining the composite decorations. It is thus enough to prove that the
  decorations of $\square f + \square g$ and $\square(f+g)$ are the same up to
  the isomorphism $p\maps G(\overline{AN} +\overline{AM}) \to
  G\overline{A(N+M)}$ between their apices, as defined in the diagram
  (\ref{diag.natural}) in the proof of Lemma \ref{lem.corelfunmonoidal}.

  This comes down to proving the following diagram commutes:
  \[
    \xymatrixrowsep{.8pc}
    \xymatrixcolsep{.8pc}
    \xymatrix{
      &&&& 
      GAN \times GAM \ar[rrdd]^{\gamma} \ar[rr]^{Gm \times Gm} && 
      G\overline{AN} \times G\overline{AM} \ar[rr]^{\gamma} && 
      G(\overline{AN}+\overline{AM}) \ar[dddd]^{Gp}_\sim \\ 
      &&&&&& 
      \textrm{\tiny(G)} \\
      1 \ar[rr]^(.4){\langle s,t \rangle} && 
      FN\times FM \ar[uurr]^{\theta} \ar[ddrr]_{\varphi} && 
      \textrm{\tiny(T)} && 
      G(AN+AM) \ar[uurr]^{G(m+m)} \ar[dd]_{G\kappa} \\
      &&&&&&& 
      \textrm{\tiny(\#\#)}\\ 
      &&&& 
      F(N+M) \ar[rr]_{\theta} && 
      GA(N+M) \ar[rr]_{Gm} && 
      G\overline{A(N+M)}
    }
  \]
  It is straightforward to check this commutes: (T) by the monoidality of
  $\theta$, (G) by the monoidality of $G$, and (\#\#) as it is the $G$-image of
  the rightmost square in (\ref{diag.natural}).
\end{proof}

%  \[
%    \xymatrixrowsep{1pc}
%    \xymatrixcolsep{1pc}
%    \xymatrix{
%      &&&& F(N+M) \ar[dd]^{F((m_N +m_M)^\opp)} \ar[rr]^{F[j_N,j_M]} 
%      && F(N+_YM) \ar[dd]^{F((m_N+_Ym_M)^\opp)} \ar[rr]^{F((m_{N+_YM})^\opp)} 
%      && F(\overline{N+_YM}) \ar[dd]^{Fn}_\sim \\ 
%      1 \ar[rr]^(.4){(s\times t)\circ\lambda^{-1}} 
%      && FN\times FM \ar[urr]^{\varphi} \ar[dr]_{Fm_N^\opp\times Fm_M^\opp} &
%      \qquad\textrm{\tiny(I)}% && \textrm{\tiny(F)} && \textrm{\tiny(C)}
%      \\ 
%      &&& F\overline{N} \times F\overline{M} \ar[r]_{\varphi} 
%      & F(\overline{N}+\overline{M})
%      \ar[rr]_{F[j_{\overline{N}},j_{\overline{M}}]} 
%      && F(\overline{N}+_Y\overline{M})
%      \ar[rr]_{F((m_{\overline{N}+_Y\overline{M}})^\opp)} 
%      && F(\overline{\overline{N}+_Y\overline{M}})
%    }
%  \]
%  Here $n$ is the isomorphism from the proof of Proposition
%  \ref{prop.corelfunctors}.
%  The leftmost square (I) commutes by the naturality of $\varphi$, the central
%  square commutes as it is the $F$-image of a pushout square in $\mathcal C$,
%  and rightmost square commutes as it is the $F$-image of the rightmost square
%  in the commutative diagram (\ref{diag.eparts}) in $\mathcal M^\opp$.
  


\begin{corollary}
  Let $\mathcal C$ be a category with finite colimits, and let $(\mathcal E,
  \mathcal M)$ be a factorisation system on $\mathcal C$. Suppose that we also
  have a lax monoidal functor
  \[
    F: (\mathcal C;\mathcal M^\opp,+) \longrightarrow (\Set, \times).
  \]
  Then we may define a category $F\mathrm{Corel}$ with objects the objects of
  $\mathcal C$ and morphisms isomorpism classes of $F$-decorated corelations.

  Write also $F$ for the restriction of $F$ to the wide subcategory $\mathcal
  C$ of $\mathcal C;\mathcal M^\opp$. We can thus also obtain the category
  $F\mathrm{Cospan}$ of
  $F$-decorated cospans. We moreover have a functor 
  \[
    F\mathrm{Cospan} \to F\mathrm{Corel}
  \]
  which takes each object of $F\mathrm{Cospan}$ to itself as an object of
  $F\mathrm{Corel}$, and each decorated cospan
  \[
    \left(
    \begin{aligned}
      \xymatrix{
	& N \\  
	X \ar[ur]^{i} && Y \ar[ul]_{o}
      }
    \end{aligned}
    ,
    \qquad
    \begin{aligned}
      \xymatrix{
	FN \\
	1 \ar[u]_{s}
      }
    \end{aligned}
    \right)
  \]  
  to its jointly-$\mathcal E$-part
  \[
    \xymatrix{
      & \overline{N} \\  
      X \ar[ur]^{i} && Y \ar[ul]_{o}
    }
  \]
  decorated by the composite
  \[
    \xymatrix{
      1 \ar[r]^s & FN \ar[r]^{Fm_N^\opp} & F\overline{N}.
    }
  \]
\end{corollary}


\section{All hypergraph categories are decorated corelation categories}
structured categories are algebras over their graphical calculus operad
\cite{SSR}. These equivalences are 2-categorical.

Write $\int$ for the Grothendieck construction. The Grothendieck construction
is

\begin{theorem}
hypergraph categories are symmetric lax monoidal functors cospan to Set.
\[
  \mathrm{HypCat} \cong \int^{\mc O \in \Set}
  \mathrm{SymLaxMon}(\cospan(\FinSet_{\mc O}), \Set)
\]
\end{theorem}
\cite{SV}
\begin{remark}
Not all hypergraph categories are decorated \emph{cospan} categories. To see
this, we can count morphisms. The possible apices and decorations are the same
for all morphisms. So for a decorated cospan category over the prop of finite
sets, the number of morphisms $0 \to 1$ cannot be more than countably many times
those $0 \to 0$ (we just get to choose an element of the apex). But the skeletal
category of vector spaces over $\mathbb{R}$ with monoidal product the tensor
product has $\mathbb{R}$ morphisms $0 \to 0$, and $\mathbb{R}^2$ morphisms $0
\to 1$.
\end{remark}

Decorated corelation categories, however, are more powerful. We can recover all
hypergraph categories by forcing the decorations to be on the coproduct of the
domain and codomain itself. For this we use the isomorphism--morphism
factorisation. Let $\mc H$ be a hypergraph category, and let $\mc C$ be the wide
subcategory of all Frobenius morphisms. Then $\mc H$ can be recovered as the
$(\mc I_{\mc C},\mc C)$-corelations decorated by global sections in $\mc H$.

\subsection{The global sections construction.}

\begin{theorem}
  All hypergraph categories are decorated corelation categories.
\end{theorem}
\begin{proof}
  Let $\mc H$ be a hypergraph category. Without loss of generality we can assume
  $\mc H$ has objects a free monoid under the tensor product; write $\mc O$ for
  a collection of generators for this free monoid, and $\FinSet_{\mc O}$ for
  $\mc O$ labelled finite sets (ie an object is a finite set $X$ together with a
  function $X \to \mc O$). This is a finitely cocomplete category. Equivalent to
  finite lists of objects in $\mc H$. 

  Define the global sections functor 
  \begin{align*}
    G\maps \cospan(\FinSet_{\mc O}) &\longrightarrow \Set \\
    A &\longmapsto \mc H(I,A) \\
    \stackrel{f}{\to}\stackrel{g}{\leftarrow} &\longmapsto \textrm{action of
    Frobenius maps}
  \end{align*}
  This is symmetric lax monoidal functor. Note that $\mc H(I,A)$ depends on the
  order we choose to convert a multiset into an object $A$ of $\mc H$.
  Nonetheless, from any two choices $A$, $A'$ we get a canonical map $A \to A'$.
  This is really that clique in $\Set$.

  Consider the category $\FinSet_{\mc O}$ with an
  (isomorphism,morphism)-factorisation system. We get a decorated corelations
  category with objects multisets of generating objects of $\mc H$, and morphisms $A \to
  B$ trivial corelations $A \to A+B \leftarrow B$ decorated by some morphism
  $s \in \mc H(I,A+B)$. Recall that this decorated corelation is only specified
  up to isomorphism; in the following we always choose representatives such that
  the apex of the jointly-isomorphic cospan is always of the form $A+B$ for
  morphisms $A \to B$.
  
  Given morphisms $s \in \mc H(I,A+B)$ and $t \in \mc H(I,B+C)$ of types $A \to
  B$ and $B \to C$ in $G\mathrm{Corel}$, composition is given by the map
  $H(I,A+B+B+C) \to H(I,A+C)$ arising as the $G$-image of the cospan $A+B+B+C
  \stackrel{[j,j]}\rightarrow A+B+C \stackrel{m}\leftarrow A+C$ where maps come
  from the pushout square
  \[
    \xymatrix{
      && A+C \ar[d] \\
      && A+B+C \\
      & A+B \ar[ur] && B+C \ar[ul] \\
      A \ar[ur] && B \ar[ul]\ar[ur] && C \ar[ul]
    }
  \]
  
  In terms of string diagrams in $\mc H$, this means composing the maps 
  \[
    \tikzset{every path/.style={line width=1.1pt}}
\begin{tikzpicture}
	\begin{pgfonlayer}{nodelayer}
		\node [style=none] (0) at (-0.25, 0.375) {};
		\node [style=none] (1) at (0.5, 0.375) {};
		\node [style=none] (2) at (-0.25, -0.375) {};
		\node [style=none] (3) at (0.5, -0.375) {};
		\node [style=none] (4) at (0.5, 0.25) {};
		\node [style=none] (5) at (0.5, -0.25) {};
		\node [style=none] (6) at (1.25, 0.25) {};
		\node [style=none] (7) at (1.25, -0.25) {};
		\node [style=none] (8) at (0.125, -0) {$s$};
		\node [style=none] (9) at (1.5, 0.25) {$A$};
		\node [style=none] (10) at (1.5, -0.25) {$B$};
	\end{pgfonlayer}
	\begin{pgfonlayer}{edgelayer}
		\draw (0.center) to (1.center);
		\draw (1.center) to (3.center);
		\draw (3.center) to (2.center);
		\draw (2.center) to (0.center);
		\draw (4.center) to (6.center);
		\draw (5.center) to (7.center);
	\end{pgfonlayer}
\end{tikzpicture}
\qquad 
\qquad
\begin{tikzpicture}
	\begin{pgfonlayer}{nodelayer}
		\node [style=none] (0) at (-0.25, 0.375) {};
		\node [style=none] (1) at (0.5, 0.375) {};
		\node [style=none] (2) at (-0.25, -0.375) {};
		\node [style=none] (3) at (0.5, -0.375) {};
		\node [style=none] (4) at (0.5, 0.25) {};
		\node [style=none] (5) at (0.5, -0.25) {};
		\node [style=none] (6) at (1.25, 0.25) {};
		\node [style=none] (7) at (1.25, -0.25) {};
		\node [style=none] (8) at (0.125, -0) {$t$};
		\node [style=none] (9) at (1.5, 0.25) {$B$};
		\node [style=none] (10) at (1.5, -0.25) {$C$};
	\end{pgfonlayer}
	\begin{pgfonlayer}{edgelayer}
		\draw (0.center) to (1.center);
		\draw (1.center) to (3.center);
		\draw (3.center) to (2.center);
		\draw (2.center) to (0.center);
		\draw (4.center) to (6.center);
		\draw (5.center) to (7.center);
	\end{pgfonlayer}
\end{tikzpicture}
  \]
  with the Frobenius map
  \[
    \tikzset{every path/.style={line width=1.1pt}}
    \begin{aligned}
\begin{tikzpicture}
	\begin{pgfonlayer}{nodelayer}
		\node [style=none] (0) at (-0.125, 0.75) {};
		\node [style=none] (1) at (-0.125, 0.25) {};
		\node [style=none] (2) at (-0.125, -0.25) {};
		\node [style=none] (3) at (-0.125, -0.75) {};
		\node [style=none] (4) at (0.5, -0) {};
		\node [style=none] (5) at (1, 0.75) {};
		\node [style=none] (6) at (1, -0.75) {};
		\node [style=none] (8) at (-0.375, 0.75) {$A$};
		\node [style=none] (9) at (-0.375, 0.25) {$B$};
		\node [style=none] (10) at (-0.375, -0.25) {$B$};
		\node [style=none] (11) at (-0.375, -0.75) {$C$};
		\node [style=none] (12) at (1.25, 0.75) {$A$};
		\node [style=none] (13) at (1.25, -0.75) {$C$};
	\end{pgfonlayer}
	\begin{pgfonlayer}{edgelayer}
		\draw (0.center) to (5.center);
		\draw [in=90, out=0, looseness=1.00] (1.center) to (4.center);
		\draw [in=-90, out=0, looseness=1.00] (2.center) to (4.center);
		\draw (3.center) to (6.center);
	\end{pgfonlayer}
\end{tikzpicture}
\end{aligned}
\quad
=
\quad
\begin{aligned}
\begin{tikzpicture}
	\begin{pgfonlayer}{nodelayer}
		\node [style=none] (0) at (-0.125, 0.75) {};
		\node [style=none] (1) at (-0.125, 0.25) {};
		\node [style=none] (2) at (-0.125, -0.25) {};
		\node [style=none] (3) at (-0.125, -0.75) {};
		\node [style=none] (4) at (0.5, -0) {};
		\node [style=none] (5) at (1.25, 0.75) {};
		\node [style=none] (6) at (1.25, -0.75) {};
		\node [style=circ2] (7) at (1, -0) {};
		\node [style=none] (8) at (-0.375, 0.75) {$A$};
		\node [style=none] (9) at (-0.375, 0.25) {$B$};
		\node [style=none] (10) at (-0.375, -0.25) {$B$};
		\node [style=none] (11) at (-0.375, -0.75) {$C$};
		\node [style=none] (12) at (1.5, 0.75) {$A$};
		\node [style=none] (13) at (1.5, -0.75) {$C$};
	\end{pgfonlayer}
	\begin{pgfonlayer}{edgelayer}
		\draw (0.center) to (5.center);
		\draw [in=90, out=0, looseness=1.00] (1.center) to (4.center);
		\draw [in=-90, out=0, looseness=1.00] (2.center) to (4.center);
		\draw (3.center) to (6.center);
		\draw (4.center) to (7.center);
	\end{pgfonlayer}
\end{tikzpicture}
\end{aligned}
  \]
  to get 
  \[
    \tikzset{every path/.style={line width=1.1pt}}
    \begin{aligned}
\begin{tikzpicture}
	\begin{pgfonlayer}{nodelayer}
		\node [style=none] (0) at (-0.25, 0.375) {};
		\node [style=none] (1) at (0.5, 0.375) {};
		\node [style=none] (2) at (-0.25, -0.375) {};
		\node [style=none] (3) at (0.5, -0.375) {};
		\node [style=none] (4) at (0.5, 0.25) {};
		\node [style=none] (5) at (0.5, -0.25) {};
		\node [style=none] (6) at (1.25, 0.25) {};
		\node [style=none] (7) at (1.25, -0.25) {};
		\node [style=none] (8) at (0.125, -0) {$t\circ s$};
		\node [style=none] (9) at (1.5, 0.25) {$A$};
		\node [style=none] (10) at (1.5, -0.25) {$C$};
	\end{pgfonlayer}
	\begin{pgfonlayer}{edgelayer}
		\draw (0.center) to (1.center);
		\draw (1.center) to (3.center);
		\draw (3.center) to (2.center);
		\draw (2.center) to (0.center);
		\draw (4.center) to (6.center);
		\draw (5.center) to (7.center);
	\end{pgfonlayer}
\end{tikzpicture}
\end{aligned}
\quad = 
\quad
\begin{aligned}
\begin{tikzpicture}
	\begin{pgfonlayer}{nodelayer}
		\node [style=none] (0) at (-0.875, 0.875) {};
		\node [style=none] (1) at (-0.125, 0.875) {};
		\node [style=none] (2) at (-0.125, 0.125) {};
		\node [style=none] (3) at (-0.875, 0.125) {};
		\node [style=none] (4) at (-0.875, -0.125) {};
		\node [style=none] (5) at (-0.125, -0.125) {};
		\node [style=none] (6) at (-0.125, -0.875) {};
		\node [style=none] (7) at (-0.875, -0.875) {};
		\node [style=none] (8) at (-0.125, 0.75) {};
		\node [style=none] (9) at (-0.125, 0.25) {};
		\node [style=none] (10) at (-0.125, -0.25) {};
		\node [style=none] (11) at (-0.125, -0.75) {};
		\node [style=none] (12) at (0.5, -0) {};
		\node [style=none] (13) at (0.75, 0.75) {};
		\node [style=none] (14) at (0.75, -0.75) {};
		\node [style=none] (15) at (-0.5, 0.5) {$s$};
		\node [style=none] (16) at (-0.5, -0.5) {$t$};
		\node [style=none] (17) at (1, 0.75) {$A$};
		\node [style=none] (18) at (1, -0.75) {$C$};
	\end{pgfonlayer}
	\begin{pgfonlayer}{edgelayer}
		\draw (0.center) to (1.center);
		\draw (1.center) to (2.center);
		\draw (2.center) to (3.center);
		\draw (3.center) to (0.center);
		\draw (4.center) to (5.center);
		\draw (5.center) to (6.center);
		\draw (6.center) to (7.center);
		\draw (7.center) to (4.center);
		\draw (8.center) to (13.center);
		\draw [in=90, out=0, looseness=1.00] (9.center) to (12.center);
		\draw [in=-90, out=0, looseness=1.00] (10.center) to (12.center);
		\draw (11.center) to (14.center);
	\end{pgfonlayer}
\end{tikzpicture}
\end{aligned}
\]
in $\mc H(I,A+C)$.
  
The monoidal product is given by
  \[
    \tikzset{every path/.style={line width=1.1pt}}
    \begin{aligned}
\begin{tikzpicture}
	\begin{pgfonlayer}{nodelayer}
		\node [style=none] (0) at (-0.25, 0.375) {};
		\node [style=none] (1) at (0.5, 0.375) {};
		\node [style=none] (2) at (-0.25, -0.375) {};
		\node [style=none] (3) at (0.5, -0.375) {};
		\node [style=none] (4) at (0.5, 0.25) {};
		\node [style=none] (5) at (0.5, -0.25) {};
		\node [style=none] (6) at (1.25, 0.25) {};
		\node [style=none] (7) at (1.25, -0.25) {};
		\node [style=none] (8) at (0.125, -0) {$s$};
		\node [style=none] (9) at (1.5, 0.25) {$A$};
		\node [style=none] (10) at (1.5, -0.25) {$B$};
	\end{pgfonlayer}
	\begin{pgfonlayer}{edgelayer}
		\draw (0.center) to (1.center);
		\draw (1.center) to (3.center);
		\draw (3.center) to (2.center);
		\draw (2.center) to (0.center);
		\draw (4.center) to (6.center);
		\draw (5.center) to (7.center);
	\end{pgfonlayer}
\end{tikzpicture}
\end{aligned}
\quad 
+
\quad
\begin{aligned}
\begin{tikzpicture}
	\begin{pgfonlayer}{nodelayer}
		\node [style=none] (0) at (-0.25, 0.375) {};
		\node [style=none] (1) at (0.5, 0.375) {};
		\node [style=none] (2) at (-0.25, -0.375) {};
		\node [style=none] (3) at (0.5, -0.375) {};
		\node [style=none] (4) at (0.5, 0.25) {};
		\node [style=none] (5) at (0.5, -0.25) {};
		\node [style=none] (6) at (1.25, 0.25) {};
		\node [style=none] (7) at (1.25, -0.25) {};
		\node [style=none] (8) at (0.125, -0) {$t$};
		\node [style=none] (9) at (1.5, 0.25) {$C$};
		\node [style=none] (10) at (1.5, -0.25) {$D$};
	\end{pgfonlayer}
	\begin{pgfonlayer}{edgelayer}
		\draw (0.center) to (1.center);
		\draw (1.center) to (3.center);
		\draw (3.center) to (2.center);
		\draw (2.center) to (0.center);
		\draw (4.center) to (6.center);
		\draw (5.center) to (7.center);
	\end{pgfonlayer}
\end{tikzpicture}
\end{aligned}
\quad = \quad 
\begin{aligned}
\begin{tikzpicture}
	\begin{pgfonlayer}{nodelayer}
		\node [style=none] (0) at (-0.875, 0.875) {};
		\node [style=none] (1) at (-0.125, 0.875) {};
		\node [style=none] (2) at (-0.125, 0.125) {};
		\node [style=none] (3) at (-0.875, 0.125) {};
		\node [style=none] (4) at (-0.875, -0.125) {};
		\node [style=none] (5) at (-0.125, -0.125) {};
		\node [style=none] (6) at (-0.125, -0.875) {};
		\node [style=none] (7) at (-0.875, -0.875) {};
		\node [style=none] (8) at (-0.125, 0.75) {};
		\node [style=none] (9) at (-0.125, 0.25) {};
		\node [style=none] (10) at (-0.125, -0.25) {};
		\node [style=none] (11) at (-0.125, -0.75) {};
		\node [style=none] (12) at (1, 0.75) {};
		\node [style=none] (13) at (1, 0.25) {};
		\node [style=none] (14) at (1, -0.25) {};
		\node [style=none] (15) at (1, -0.75) {};
		\node [style=none] (16) at (-0.5, 0.5) {$s$};
		\node [style=none] (17) at (-0.5, -0.5) {$t$};
		\node [style=none] (18) at (1.25, 0.75) {$A$};
		\node [style=none] (19) at (1.25, 0.25) {$C$};
		\node [style=none] (20) at (1.25, -0.25) {$B$};
		\node [style=none] (21) at (1.25, -0.75) {$D$};
	\end{pgfonlayer}
	\begin{pgfonlayer}{edgelayer}
		\draw (0.center) to (1.center);
		\draw (1.center) to (2.center);
		\draw (2.center) to (3.center);
		\draw (3.center) to (0.center);
		\draw (4.center) to (5.center);
		\draw (5.center) to (6.center);
		\draw (6.center) to (7.center);
		\draw (7.center) to (4.center);
		\draw (8.center) to (12.center);
		\draw [in=180, out=0, looseness=1.00] (9.center) to (14.center);
		\draw [in=180, out=0, looseness=1.00] (10.center) to (13.center);
		\draw (11.center) to (15.center);
	\end{pgfonlayer}
\end{tikzpicture}
\end{aligned}
\]
recalling that we have chosen to represent the equivalence class of corelations
$A+C \to B+D$ with the apex $A+C+B+D$.

  Taking a hint from the compact closed structure, it is straightforward to
  construct a pair of inverse hypergraph functors between $G\mathrm{Corel}$ to $\mc H$.
  Indeed, $G\mathrm{Corel}$ is constructed to have the same collection of
  objects as $\mc H$; simply have the functors be the `identity' on objects. On
  morphisms, we take $f: A \to B$ in $\mc H$ to its `name' $\hat f: I \to A+B$
  as a morphism of $G\mathrm{Corel}$. This is a bijection. 
  
  To check it is composition and monoidal product preserving, we can easily use
  diagrammatic reasoning. For example

  Therefore $\mc H$ is isomorphic
  as a hypergraph category to $G\mathrm{Corel}$. 
\end{proof}

\begin{theorem}
  All hypergraph functors are decorated corelation functors.
\end{theorem}
\begin{proof}
  Let $\mc H$ and $\mc H'$ be hypergraph categories, and $T\maps \mc H \to \mc
  H'$ be a hypergraph functor. By the above theorem, there exist symmetric lax
  monoidal functors 
  \[
    G\maps \cospan(\FinSet_{\mc O_{\mc H}}) \to \Set 
  \]
  and 
  \[
    G'\maps \cospan(\FinSet_{\mc O_{\mc H'}}) \to \Set 
  \]
  such that $\mc H = G\mathrm{Corel}$ and $\mc H' = G'\mathrm{Corel}$.
  Furthermore, define a functor $A\maps \FinSet_{\mc O_{\mc H}} \to \FinSet_{\mc
    O_{\mc H'}}$ taking $N \to \mc O_{\mc H}$ to $N \to \mc O_{\mc H} \to \mc
  O_{\mc H'}$, where the second map is that by the functor $T$ on objects of
  $\mc H$. We claim this is a well-defined colimit-preserving functor and show
  that $T$ can be constructed from a monoidal natural transformation between $G$
  and $G'\circ A$.
\end{proof}

Compare with Spivak Vagner construction.

\subsection{Examples.}
We give some examples reproducing hypergraph categories as decorated corelations
categories.

\begin{example}
  Example: empty decorations and equivalence relations. 
  
  Consider the hypergraph category $\cospan(\FinSet)$. This is the simplest
  hypergraph category: it is free hypergraph category on the one object discrete
  category. We show how to recover it as a decorated corelation category.

  As per Example \ref{ex.undeccorel}, $\cospan(\FinSet)$ is the hypergraph
  category of undecorated (morphism-isomorphism)-corelations in $\FinSet$. It is
  also the partition-decorated (isomorphism-morphism)-corelations in $\FinSet$.   
  
  First, the global sections functor $G\maps \cospan(\FinSet) \to \Set$ takes
  each finite set $X$ to the set of (equivalence classes of) cospans $0 \to D
  \leftarrow X$; that is, to the set of functions $X \to D$ where a unique $D$
  is chosen for each finite cardinality. Given a cospan $X \stackrel{f}\to N
  \stackrel{g}\leftarrow Y$, its image under the global sections functor maps a
  function $a\maps X \to D$, to the function $Y \to N+_YD$ given by
  \[
    \xymatrix{
      & X \ar[r]^d \ar[d]_f & D \ar[d] \\
      Y \ar[r]^g  & N \ar[r] & N+_YD
    }
  \]
  where the square is a pushout square.

  The coherence maps $\gamma_1\maps 1 \to G\varnothing$ map the unique element
  of $1$ to the unique function $!\maps \varnothing \to \varnothing$, and
  $\gamma_{X,Y}$ maps a pair of functions $a\maps X \to D$, $b\maps Y \to E$ to
  $a+b\maps X+Y \to D+E$. This is a symmetric lax monoidal functor. We use this
  functor to decorate cospans in $\FinSet$. 

  A decorated cospan in $\FinSet$ with respect to this functor is a cospan $X
  \to N \leftarrow Y$ in $\FinSet$ together with a function $N \to D$ for some
  finite set. Using the isomorphism--morphism factorisation, a decorated
  corelation (a morphism in $G\mathrm{Corel}$) is a
  cospan $X \to X+Y \leftarrow Y$, together with a function $X+Y \to D$.
  This is the same as a cospan. 

  The hypergraph structure is given by the decoration $X+X \to X$ etc, as the
  shift from cospans to corelations takes the `factored out part' and puts it
  into the decoration. At this point the morphisms are specified entirely by
  their decoration.

  It is straightforward to show the two categories are isomorphic. Note that the
  identity on $\FinSet$ maps $\mc I_\FinSet$ into $\FinSet$, and so extends to
  morphism $\FinSet \to \cospan(\FinSet)$. We can define a monoidal natural
  transformation $1(X) =1 \stackrel{\theta_X}\to GX = \{X \to D\}$ mapping the
  unique element to the identity function $1_X\maps X \to X$.
  \[
    \xymatrixrowsep{2ex}
    \xymatrix{
      \FinSet = \FinSet; \mc I_{\FinSet}^\opp \ar[dd]_{\iota} \ar[drr]^1  \\
      &\twocell \omit{_\:\theta}& \Set \\
      \cospan(\FinSet) = \FinSet; \FinSet^\opp \ar[urr]_{G} 
    }
  \]
  It is easy to verify that this is a monoidal natural transformation. This
  gives the hypergraph functor we expect, mapping the undecorated cospan $X \to
  N \leftarrow Y$ to $X \to X+Y \leftarrow Y$ decorated by $X+Y \to N$. 

  Like so many examples before, it is easy to verify this is a full, faithful,
  bijective-on-objects hypergraph functor.
\end{example}

\begin{example}
  The previous example extends to any finitely cocomplete category $\mc C$: the
  hypergraph category $\cospan(\mc C)$ can always be constructed as (i)
  trivially decorated $(\mc C, \mc I_{\mc C})$-corelations, or (ii) $(\mc I_{\mc
  C}, \mc C)$-corelations decorated by equivalence classes of morphisms with
  domain the apex of the corelation, and moreover the isomorphism of these
  hypergraph categories is given by a monoidal natural transformation between
  the decorating functors.

  More general still, a category of trivially decorated $(\mc E, \mc
  M)$-corelations in $\mc C$ can always be constructed also as $(\mc I_{\mc C},
  \mc C)$-corelations decorated by equivalence classes of morphisms in $\mc E$
  with domain the apex of the corelation, and the isomorphism of these
  hypergraph categories is given by a monoidal natural transformation between
  the decorating functors.

  Most general, the theorem implies that any category of decorated
  corelations can be constructed also as $(\mc I_{\mc C}, \mc C)$-corelations
  decorated by codomain decorated morphisms in $\mc E$. The latter are specified
  by a functor $\cospan(\mc C) \to \Set$, as in the theorem.

  The latter form is good for constructing functors that have image the
  corelation category.
\end{example}

Note that with decorated corelations we have a method of constructing hypergraph
categories from other hypergraph categories. This is black boxing. 

\section{Examples} \label{sec:excor}
\subsection{Path integrals and matrices}

Let $R$ be a commutative ring. Take functor
\begin{align*}
  R^{(-)}: (\mathrm{FinSet}^\opp,+) &\longrightarrow (\Set,\times) \\
  X &\longmapsto R^X
\end{align*}

Then $R^{(-)}\mathrm{Cospan}$ is path integrals, $R^{(-)}\mathrm{Corel}$ is
matrices over $R$. 

Many aspects of this example are `atypical', regarding the intuition we have
been working towards.
Note that the monoidal product here is the tensor product of
matrices, not the biproduct. Indeed, there is no special commutative Frobenius
algebra in $\Vect$ if we use the biproduct, but if we use the tensor product
then these correspond to orthonormal bases (Vicary). The comultiplication is the
diagonal map, multiplication is codiagonal. unit produces basis.

We note that you could take decorations here in the category $R\mathrm{Mod}$ of
$R$-modules. While Proposition \ref{prop.setdecorations} shows that the
resulting decorated cospans category would be isomorphic, this hints at an
enriched version of the theory.

\subsection{Two constructions for linear relations}

We saw earlier that linear relations are epi-mono corelations in $\Vect$. The
hypergraph structure is given by addition. We show how to recover this in
another construction. We also get a hypergraph functor between them. This is
very useful for compositional linear relations semantics of diagrams.

We can also construct linear relations in $\Vect^\opp$.

\begin{align*}
  \maps\mathrm{Cospan}(\mathrm{FinSet}) &\longrightarrow \mathrm{Set} \\
  X &\longmapsto \{\mathrm{subspaces of }k^X\} \\
  f: X \to Y &\longmapsto L \mapsto \{v \mid v\circ f \in L\} \\
  f^\opp: X \to Y &\longmapsto L \mapsto \{v = u \circ f \mid u \in L\}
\end{align*}

Then $\mathrm{Cospan}$ is cospans decorated by subspaces, and $\mathrm{Corel}$
is linear relations. This is important for circuits work \cite{BF,BSZ}.

\subsection{Automata}
This construction comes immediately from Walters et al. Automata are alphabet
labelled graphs. There is a decorated cospan functor to categories enriched over
languages, and this factors nicely to get a decorated corelation category with
morphisms languages recognised between points in domain and codomain.







%  Our proof will have an inductive flavour. We first assume $\corel_(\mc C)$ is
%  well-defined as a hypergraph category, and show  We caution that we still have not proved that $\corel(\mc C)$ is a category,
%  let alone a hypergraph category. Thus to begin, all we can show is that the map
%  $\square$ is composition-preserving, and then that $\square$ respects the
%  monoidal and hypergraph structure. But this is enough to prove both
%  results! Indeed, specialising to the case where $A$ is the identity functor on
%  $\mc C$ and $(\mc E,\mc M)=(\mc C, \mc I_{\mc C})$, observe   
%  \[
%    \square \maps \cospan(\mc C) \longrightarrow \corel(\mc C)
%  \]
%  is then the map taking each cospan to its jointly $\mc E$-like part. Note
%  $\square$ maps fully (surjectively-on-morphisms) and bijectively-on-objects
%  onto $\corel(\mc C)$, and by definition the coherence and hypergraph maps on
%  $\corel(\mc C)$ are precisely the image of the corresponding maps of
%  $\cospan(\mc C)$. As $\cospan(\mc C)$ is a hypergraph category and $\square$
%  is composition-preserving, we can consequently conclude that all corelation
%  categories are indeed hypergraph categories, and hence that $\square$---in the
%  general case---is a hypergraph functor.
%
%  Finally, specialising to the case where $A$
%  is the identity functor on $\mc C$ and $(\mc E,\mc M)$ is the factorisation
%  system $(\mc C, \mc I_{\mc C})$, we observe that the domain of $\square$ is
%  then $\cospan(\mc C)$, and that $\square$ maps it fully
%  (surjectively-on-morphisms) and bijectively-on-objects onto $\corel(\mc C')$.
%  This shows that all corelation categories are indeed deserving of the name
%  category and moreover hypergraph category, and hence that $\square$ is
%  deserving of the name hypergraph functor.
