\chapter{Further directions}

Open systems. 

\section{Theoretical}
\subsection{Bounded colimits}
Why are epis so important? Extended theory

Grothendieck construction (Kissinger comment)

\subsection{Classifying hypergraph categories}


\subsection{Other categorical structures}
Not all diagrammatic structures have the same hypergraph network style. For
example, some have substantive notions of input and output. Spivak's work on
operads of wiring diagrams 


\section{Applied}
While there is not time to explore the applications of the compositional
perspective on circuits developed in Chapter refSIX, we mention two
applications, and some generalisations.
\subsection{Passive linear networks}
  Inverse problems
  a type of elementary factorization in $\mathrm{Circ}$ serves as a geometric
   model for "discrete harmonic continuation."  It describes a sufficient
   condition for the size of connections through the graph to be detectable from
   the linear-algebraic properties of the boundary data. \cite{Jek}

\subsection{Open Markov processes}
Bring in notions of entropy, summary statistics. \cite{BFP16}

\subsection{Flow networks}
cf linear algebra. diagrams+rewrite rules, algebraic semantics, summary arrays.



\subsection{Chemical reaction networks}


\subsection{Automata}
Walters and the Kleene theorem as black-boxing.
