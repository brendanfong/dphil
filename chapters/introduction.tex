\phantomsection
\addcontentsline{toc}{chapter}{Introduction}
\chapter*{Introduction}

This is a thesis in the theory and applications of hypergraph categories. A
hypergraph category is 

---first defined by Carboni and Walters \cite{CW, Ca} in the context of
representation theory---

The work here is underpinned not just by philosophical precedent, but by a rich
tradition in mathematics, physics, and computer science. First and foremost, it
draws from category theory and the pioneering work of Joyal and
Street on string diagrams for monoidal categories \cite{JS91,JS93}.

and conversely monoidal categories
provide an algebraic foundation for the intuitions behind Feynman diagrams.  The
key insight is the use of categories where morphisms describe physical
processes, rather than structure-preserving maps between mathematical objects
\cite{BaezStay,CP}
 
  compositional accounts of semantics associated to topological
diagrams has long been a technique associated with topological
quantum field theory, dating back to \cite{At}.

These techniques have filtered into more immediate applications, particularly in
computation and quantum computation \cite{AC04,Ba1,Sel07}. 
Kindergarten quantum mechanics. 

Category-theoretic frameworks for general systems theory have been developed
before. Notably Goguen and Rosen led efforts. Goguen from a more computer
science perspective, Rosen more in biology.


Work on categorical approaches to control systems goes back at least to Goguen
\cite{Go} and Arbib and Manes \cite{AM}. 



Cospans are already familiar as a formalism for making entities with an
arbitrarily designated `input end' and `output end' into the morphisms of a
category.  For example, in topological quantum field theory we use special
cospans called `cobordisms' to describe pieces of spacetime \cite{BL,BaezStay}.

An advantage of the decorated cospan framework is that the resulting categories
are hypergraph categories, and the resulting functors respect this structure.
As dagger compact categories, hypergraph categories themselves have a rich
diagrammatic nature \cite{Sel11}, and in cases when our decorated cospan categories
are inspired by diagrammatic applications, the hypergraph structure provides
language to describe natural operations on our diagrams, such as juxtaposing,
rotating, and reflecting them.

At the present time, 

it has most recently had
significant influence in the nascent field of categorical network theory, with
application to automata and computation \cite{KSW2, Sp}, electrical circuits
\cite{BF}, signal flow diagrams \cite{BSZ, BE}, Markov processes \cite{BFP,
ASW}, and dynamical systems \cite{VSL}, among others. 

applying string diagrams to engineering, with the aim of
giving diverse diagram languages a unified foundation based on category theory
\cite{KSW,RSW05}. 

Baez and Erbele
\cite{BE}, Vagner, Spivak, and Lerman \cite{VSL}, as well as Bonchi,
Soboc\'inski, and Zanasi \cite{BSZ,BSZ2,BSZ3,Za}. 


Spivak operad approach

Most similar is the work of Walters, together with Sabadini and Rosebrugh. In
particular, automata.

cospans and colimits



\subsection*{Organisation of this thesis}
This thesis is divided into two parts. Part I, comprising Chapters
\ref{ch.hypcats} to \ref{ch.deccorels}, focusses on mathematical foundations. In
it we develop the theory of hypergraph categories and a powerful tool for
defining and manipulating them: decorated corelations. Part II, comprising
Chapters \ref{ch.sigflow} to \ref{ch.further}, then discusses applications of
this theory to examples of open systems.

Chapter \ref{ch.hypcats} introduces hypergraph categories. These are symmetric
monoidal categories in which every object is equipped with the structure of a
special commutative Frobenius monoid in a way compatible with the monoidal
product. As we will rely heavily on properties of monoidal categories, their
functors, and their graphical calculus, we begin with a whirlwind review of
these ideas. We then provide a definition of hypergraph categories, a
strictification theorem, and an important example: the category of cospans in a
category with finite colimits.

Cospans and their composition, using pushouts, 

Chapter \ref{ch.deccospans}

Chapter \ref{ch.corelations}

Chapter \ref{ch.deccorels}

The central message of decorated corelations is that hypergraph structure
requires some sort of uniformity in the composition rule, and it is easier to
work by acknowledging this structure, defining them as algebras over some
theory. 

Hypergraph categories are really just $\Set$-valued lax symmetric monoidal
functors: and these, being simply data structures, are often simpler to work
with.


Then applications. Chapter \ref{ch.sigflow} First to control theory from this behavioural perspective,
demonstrating the ideas of black boxing and corelations, open systems and
control by interconnection. We give a new characterisation of controllability.
Second to passive linear networks, which speaks to the idea that we can find
isomorphisms across disciplines, and explore the insights in new ways.

Chapter \ref{ch.circuits}

Chapter \ref{ch.further}


\subsection*{Statement of work}

The Examination Schools make the following request:
\begin{quote}
\emph{Where some part of the thesis is not solely the work of the candidate or
has been carried out in collaboration with one or more persons, the candidate
shall submit a clear statement of the extent of his or her own contribution.}
\end{quote}
I address this now. 

The first four chapters are my own work. The applications chapters were
developed with collaborators. 

Chapter \ref{ch.sigflow} arises from a weekly seminar with Paolo Rapisarda and
Pawe\l\ Soboc\'inski at Southampton in the Spring of 2015. The text is a minor
adaptation of that in the paper \cite{FRS16}. For that paper I developed the
corelation formalism, providing a first draft. Pawe\l\ provided much expertise
in signal flow graphs, significantly revising the text and contributing the
section on operational semantics. Paolo contributed comparisons to classical
methods in control theory.  A number of anonymous referees contributed helpful
and detailed comments.

Chapter \ref{ch.circuits} is joint work with John Baez; the majority of the text
is taken from our paper \cite{BF}. For that paper John supplied writing on
Dirichlet forms and the principle of minimum power that became the second
section of Chapter \ref{ch.circuits}, as well as parts of the next two sections.
I produced a first draft of the rest of the paper. We collaboratively revised
the text for publication.


%We leave a precise definition of hypergraph categories to Chapter
%\ref{ch.hypcats}.
%
%Thus by algebra in our formal statement of the principle of compositionality we
%mean hypergraph category, and by homomorphism we mean hypergraph functor.

%\begin{quotation}
%  While in the past, science tried to explain observable phenomena by reducing
%  them to an interplay of elementary units investigable independently of each
%  other, conceptions appear in contemporary science that are concerned with what
%  is somewhat vaguely termed `wholeness', i.e. problems of organization,
%  phenomena not resolvable into local events, dynamic interactions manifest in
%  difference of behaviour of parts when isolated or in a higher configuration,
%  etc.; in short, `systems' of various order not understandable by investigation
%  of their respective parts in isolation. 
%\end{quotation}
%
%%http://vhpark.hyperbody.nl/images/a/aa/Bertalanffy-The_Theory_of_Open_Systems_in_Physics_and_Biology.pdf
%
%This dissertation represents the beginnings of an attempt at a category
%theoretic framework for general systems theory. A loosely organised body of
%research dating back to biologist Ludwig von Bertalanffy in the mid-20th
%century \cite{Ber}, general systems theory represents the study of systems built
%from rich interconnections of simple component systems. Our central example in
%this dissertation will be passive linear circuits: systems built from linear
%resistors, inductors, and capacitors.  While each component here is simple,
%networks built from such components are complex enough to form the foundation of
%modern electronics.
%
%Feedback; Wiener's cybernetics was often viewed as identical in agenda, 
%
%More technical progress has been realised: cybernetics, catastrophe theory,
%chaos theory, complex systems, network analysis
%
%The goal here is to investigate these ideas from an algebraic perspective,
%creating structural and compositional techniques for modelling
%interconnection, open systems, and formal analogies, or isomorphisms, between
%different fields.
%

%
%\subsection*{What do we want a theory of open systems to look like?}
%
%Now that we have explored the algebra and interconnection, what do we mean by
%system? We might be tempted to simply develop structures for interconnection,
%and say these model whatever they end up modelling. I hope I have made a case
%that the application might be broad. But it falls on me to also show that, in
%practice, that the applications do exist. For this we take the following,
%definition of system, inspired especially by Willems.
%

%This vision extended into the social sciences and humanities, influencing
%economics, urban planning, sociology, psychology, management, philosophy. Eg
%Forrester. These aspects of system theory are beyond the scope of the present
%work.
%
%
%We capture this syntactic conception of a system is formally through the notion
%of a hypergraph category---a symmetric monoidal category in which every object
%is equipped with a special commutative Frobenius monoid in a way compatible with
%the monoidal product. This framework provides precise language for describing
%how we manipulate and interact with systems.
%


%Broad appeal. It has always had a unification bent to it \cite{Ber50}
%
%In this program Odum, energese. diagrams. universal language SBGN, UML
%\begin{quote}
%  Not only are general aspects and viewpoints alike in different sciences;
%  frequently we find formally identical or isomorphic laws in different fields.
%  In many cases, isomorphic laws hold for certain classes or subclasses of
%  `systems', irrespective of the nature of the entities involved. There appear
%  to exist general system laws which apply to any system of a certain type,
%  irrespective if the particular properties of the system and of the elements
%  involved.
%\end{quote}
%
%what is the algebra of network languages?
%
%hypergraph categories.
%
%graph example
%
%black-boxing, information compression.
%
%The Yoneda lemma and representability.
%
%traditional notions of composition often rely on functional composition.
%
%Example: trijunction


%    The homomorphism from syntax to semantics (ie. principle of compositionality)
%then implies we can substitute equivalent components to get equivalent systems
%
%In the spirit of prioritising the relationships between objects---reminiscent of
%the role of measurements of quantum mechanics or Yoneda's lemma in category
%theory---we wish to consider two syntactic expressions $A$ and $B$ for open systems
%equivalent if their terminals are in one-to-one correspondence, and if we
%substitute one expression for the other in any . The semantics are then these equivalent classes.
%
