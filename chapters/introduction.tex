\phantomsection
\addcontentsline{toc}{chapter}{Introduction}
\chapter*{Introduction}

This is a thesis in the mathematical sciences, with emphasis on the mathematics.
But before we get to the category theory, I want to say a few, brief words about
the scientific tradition this thesis draws from.

Mathematics is the language of science. New science, new scientific paradigms,
requires new language.

This dissertation represents the beginnings of an attempt at a category
theoretic framework for general systems theory. A loosely organised body of
research dating back to biologist Ludwig von Bertalanffy in the mid-20th
century \cite{Ber}, general systems theory represents the study of systems built
from rich interconnections of simple component systems. Our central example in
this dissertation will be passive linear circuits: systems built from linear
resistors, inductors, and capacitors.  While each component here is simple,
networks built from such components are complex enough to form the foundation of
modern electronics.

General system theory seeks to talk about composition
\begin{quotation}
  While in the past, science tried to explain observable phenomena by reducing
  them to an interplay of elementary units investigable independently of each
  other, conceptions appear in contemporary science that are concerned with what
  is somewhat vaguely termed `wholeness', i.e. problems of organization,
  phenomena not resolvable into local events, dynamic interactions manifest in
  difference of behaviour of parts when isolated or in a higher configuration,
  etc.; in short, `systems' of various order not understandable by investigation
  of their respective parts in isolation. 
\end{quotation}

%http://vhpark.hyperbody.nl/images/a/aa/Bertalanffy-The_Theory_of_Open_Systems_in_Physics_and_Biology.pdf

From closed systems to open
\begin{quotation}
  Conventional physics deals only with closed systems, i.e. systems which are
  considered to be isolated from their environment.

  However, we find systems which by their very nature and definition are not
  closed systems. Every living organism is essentially an open system. It
  maintains itself in a continuous inflow and outflow, a building up and
  breaking down of components, never being, so long as it is alive, in a state
  of chemical and thermodynamic equilibrium but maintained in a so-called steady
  state which is distinct from the latter.  
\end{quotation}

It has always had a unification bent to it \cite{Ber50}

\begin{quotation}
  Not only are general aspects and viewpoints alike in different sciences;
  frequently we find formally identical or isomorphic laws in different fields.
  In many cases, isomorphic laws hold for certain classes or subclasses of
  'systems', irrespective of the nature of the entities involved. There appear
  to exist general system laws which apply to any system of a certain type,
  irrespective if the particular properties of the system and of the elements
  involved.
\end{quotation}
We will return to these ideas about isomorphic laws in Chapter 5.

This vision extended into the social sciences and humanities, influencing
economics, urban planning, sociology, psychology, management, philosophy. Eg
Forrester. These aspects of system theory are beyond the scope of the present
work.

Part of this vision, most notably in von Bertalanffy's home field of
systems biology, has been realised through 


    ``Systems biology...is about putting together rather than taking apart,
    integration rather than reduction. It requires that we develop ways of
    thinking about integration that are as rigorous as our reductionist
    programmes, but different....It means changing our philosophy, in the full
    sense of the term'' (Denis Noble).

Feedback; Wiener's cybernetics was often viewed as identical in agenda, 

More technical progress has been realised: cybernetics, catastrophe theory,
chaos theory, complex systems, network analysis

The goal here is to investigate these ideas from an algebraic perspective,
creating structural and compositional techniques for modelling
interconnection, open systems, and formal analogies, or isomorphisms, between
different fields.

\paragraph{A diagrammatic approach.}

In this program Odum, diagrams. universal language SBGN, UML

what is the algebra of network languages?

hypergraph categories.

graph example

black-boxing, information compression.

examples examples.

Example: trijunction

signal flow

circuits

automata, markov processes, flow networks.

The Yoneda lemma and representability.


Syntactically, by system we mean a `box' with finitely many `ports' through which it
interfaces with the external world. These ports may be of different types. These
systems may be connected together, along ports of the same type, to form larger
systems. Examples of such systems abound; a motivating source of them is
network-style diagrammatic languages, such as the aforementioned electrical
circuit diagrams, but also including chemical reaction networks, Petri nets,
automata, and Markov processes.

We capture this syntactic conception of a system is formally through the notion
of a hypergraph category---a symmetric monoidal category in which every object
is equipped with a special commutative Frobenius monoid in a way compatible with
the monoidal product. This framework provides precise language for describing
how we manipulate and interact with systems.


But what \emph{is} a system, and what do these manipulations and
interconnections mean? Taking inspiration from Willems \cite{Wi} and Deutsch
\cite{D} among others, we consider a system to be merely the set of all possible
different observations one might make by measuring all relevant variables at all
ports of the `box' we use to represent it.  We call this set the
\emph{behaviour} of the system. This arises from a view of physical laws as a
mechanism for simply partitioning the set of all trajectories of a system, the
so-called \emph{universum}, into possible and impossible trajectories.

For example, consider a resistor of resistance $r$. This has two ports---the two
ends of the resistor---and at each port we may measure the potential, and the
current flowing into the port. Now the resistor is governed by Kirchhoff's
current law, or conservation of charge, and Ohm's law. Conservation of charge
states that the current flowing into one port must equal the current flowing out
of the other port, while Ohm's law states that this current will be proportional
to the potential difference, with constant of proportionality $1/r$. Thus the
behaviour of the resistor is the set 
\[
  \big\{\big(\phi_1,\phi_2,
    -\tfrac1r(\phi_2-\phi_1),\tfrac1r(\phi_2-\phi_1)\big)\,\big\vert\,
    \phi_1,\phi_2 \in \mathbb{R}\big\}.
\]
Here the universum is the set
$\mathbb{R}\oplus\mathbb{R}\oplus\mathbb{R}\oplus\mathbb{R}$, where the
summands represent respectively the potentials and currents at each of the two
terminals.

% universum $\mathcal U$ of trajectories, behaviour $\mathcal P(\mathcal U)$,
% principle $\mathcal P(\mathcal P(\mathcal U))$,

The variables associated to each port are determined by the type of the port.
Interconnection of ports then, simply asserts the identification of the
variables at the connected ports. On the level of behaviours, this becomes a
generalised version of composition of relations.

In this thesis I argue that this general framework has wide applicability to
applied science and engineering, and with appropriate additional mathematical
tools allows one to formalise certain diagrammatic languages and their
relationships.

The central message of decorated corelations is that hypergraph structure
requires some sort of uniformity in the composition rule, and it is easier to
work by acknowledging this structure, defining them as algebras over some
theory. But we can weaken this notion of theory.

Beyond category theoretic interest, the motivation for such a method lies in
developing compositional accounts of semantics associated to topological
diagrams. While this has long been a technique associated with topological
quantum field theory, dating back to \cite{At}, it has most recently had
significant influence in the nascent field of categorical network theory, with
application to automata and computation \cite{KSW2, Sp}, electrical circuits
\cite{BF}, signal flow diagrams \cite{BSZ, BE}, Markov processes \cite{BP,
ASW}, and dynamical systems \cite{VSL}, among others. 

Cospans are already familiar as a formalism for making entities with an
arbitrarily designated `input end' and `output end' into the morphisms of a
category.  For example, in topological quantum field theory we use special
cospans called `cobordisms' to describe pieces of spacetime \cite{BL,BaezStay}.

An advantage of the decorated cospan framework is that the resulting categories
are hypergraph categories, and the resulting functors respect this structure.
As dagger compact categories, hypergraph categories themselves have a rich
diagrammatic nature \cite{Se}, and in cases when our decorated cospan categories
are inspired by diagrammatic applications, the hypergraph structure provides
language to describe natural operations on our diagrams, such as juxtaposing,
rotating, and reflecting them.

Influence of computer science. syntax semantics. rewrite rules/local rather than
global analysis.

Inspired by Coecke Abramsky CQM

\section{Outline}
Chapter 1 introduces hypergraph categories. Then decorated cospans and decorated
corelations. Colimits?

Then applications. First to control theory from this behavioural perspective,
demonstrating the ideas of black boxing and corelations, open systems and
control by interconnection. We give a new characterisation of controllability.
Second to passive linear networks, which speaks to the idea that we can find
isomorphisms across disciplines, and explore the insights in new ways.

\section{Related work}

Category-theoretic frameworks for general systems theory have been developed
before. Notably Goguen and Rosen led efforts. Goguen from a more computer
science perspective, Rosen more in biology.



Spivak 

Zanasi, Soboc\'inski, Bonchi

Most similar is the work of Walters, together with Sabadini and Rosebrugh. In
particular, automata.

\section{Contributions of collaborators}
%Where some part of the thesis is not solely the work of the candidate or has been carried out in collaboration with one or more persons, the candidate shall submit a clear statement of the extent of his or her own contribution.

The first three chapters are my own work. The applications chapters were
developed with collaborators. Chapter 4 arises from a weekly seminar with Paolo
Rapisarda and Pawe\l\ Soboc\'inski. I developed the corelation formalism and provided a
first draft of the paper. Pawe\l\ provided much expertise in signal flow graphs,
significantly revising the text and contributing the section on operational
semantics. Paolo contributed comparisons to classical methods in control theory. 
Much of the text is taken from the paper \cite{FonRapSob16}.

Chapter 5 is joint work with John Baez. John proposed the topic of research, and
supplied some notes on Ohm's law and the principle of minimum power. I
was responsible for producing the first draft of the chapter from this start.
John guided and assisted revisions of this text for publication \cite{BaeFon16}.

I thank my collaborators.
