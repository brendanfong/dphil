\phantomsection
\addcontentsline{toc}{chapter}{Non-technical overview}
\chapter*{Non-technical overview}

This section is for all the friends, family, and strangers who have ever asked
me what is it that I actually do, only to come away more confused than before.

Before we embark upon this long journey, I want to say a few words about where
we're coming from and where I hope to take us. Mathematical dissertations have a
reputation for being inpenetrable. Piper Harron shows us they 

While I will refrain from a running commentary on mathematical culture in my thesis, I'll 


How do you do research in mathematics? Isn't it all solved already?

In this thesis the main theme is the power of abstraction. How did we invent
number systems?

With civilization came trade, and with trade a need for a number system. By the 4th millennium BCE, Sumerian innovators had responded admirably to this task, and many bureaucrats were now skilled in drawing up contracts trading livestock, or sheaves of wheat, or jars of oil. But there was a problem: each community had arrived at different ways to represent numbers. So the man experienced in trading wheat had no ability to describe quantities of cows. It took another thousand years to unify these systems into single concept of abstract number.

By now this is a pattern we know well, 

The dream here is to contribute to the invention of richer reasoning systems in
the same vein.

Circuit diagrams.

Wigner once described an unreasonable effectiveness of mathematics. Biologists hardly feel the same.
Why? Reductionist themes in science and mathematics.

Networks and interconnection are better suited to biology.

Where will it go? Formal tools for reasoning about systems and their
interconnection.

Programming. Dijkstra: languages are too easy to write, not easy enough to read. Coding is 99\% debugging.

But we must take small steps towards these grand ideas. What follows is just one
small contribution.

